\appendix{Data Sets Used in Examples}

\section{PCB}

Data on the concentrations of polychlorinated biphenyl (PCB) residues in
a series of lake trout from Cayuga Lake, NY, were reported in
\citeasnoun{bach:seru:youn:lisk:1972}
and are reproduced in Table~\ref{atbl:pcb}.
%\glossary{ Bache, C.A.}
%\glossary{ Serum, J.W.}
%\glossary{ Youngs, W.D.}
%\glossary{ Lisk, D.J.}
\begin{table}
  \begin{center}
    \caption{\label{atbl:pcb}
    PCB concentration versus age for lake trout.  }
    \begin{tabular}{r r r r}\hline
      Age&PCB Conc.&Age&PCB Conc.\\
      (years)&(ppm)&(years)&(ppm)\\ \hline
      1&0.6&6&3.4\\
      1&1.6&6&9.7\\
      1&0.5&6&8.6\\
      1&1.2&7&4.0\\
      2&2.0&7&5.5\\
      2&1.3&7&10.5\\
      2&2.5&8&17.5\\
      3&2.2&8&13.4\\
      3&2.4&8&4.5\\
      3&1.2&9&30.4\\
      4&3.5&11&12.4\\
      4&4.1&12&13.4\\
      4&5.1&12&26.2\\
      5&5.7&12&7.4\\ \hline
    \end{tabular}
  \end{center}
  \begin{quote}\small
    Copyright 1972 by the AAAS.
    Reproduced from {\em SCIENCE}, 1972, {\bf 117}, 1192--1193,
    with permission of the authors.
  \end{quote}
\end{table}
The ages of the fish were accurately known, because the fish are
annually stocked as yearlings and distinctly marked as to year class.
Each whole fish was mechanically chopped, ground, and thoroughly mixed,
and 5-gram samples taken.
The samples were treated and PCB residues in parts per million (ppm)
were estimated using column chromatography.

A linear model
$$
f ( x, \bbeta )=\beta_1+\beta_2 x
$$
is proposed where $f$ is predicted ln(PCB
concentration) and $x$ is $\cubage$.

\section{Rumford}

Data on the amount of heat generated by friction were obtained by
Count Rumford in 1798.
A bore was fitted into a stationary cylinder and pressed against the
bottom by means of a screw.
The bore was turned by a team of horses for 30 minutes, after which
Rumford ``suffered the thermometer to remain in its place
nearly three quarters of an hour, observing and noting down, at small
intervals of time, the temperature indicated by it'' \cite{roll:1950}.
(See Table~\ref{atbl:rum})
\begin{table}
  \begin{center}
    \caption{\label{atbl:rum}
    Temperature versus time for Rumford cooling experiment.}
    \begin{tabular}{r r r r}
      Time&Temperature&Time&Temperature\\
      (min)&($^\circ$F)&(min)&($^\circ$F)\\
      4&126&24&115\\
      5&125&28&114\\
      7&123&31&113\\
      12&120&34&112\\
      14&119&37.5&111\\
      16&118&41&110\\
      20&116\\
    \end{tabular}
  \end{center}
  \begin{quote}\small
    Reprinted with permission from ``The Early Development of
    the Concepts of Temperature and Heat: The Rise and Decline of the
    Caloric Theory.'' by Duane Roller, Harvard University Press, 1950.
  \end{quote}
  %\glossary{ Roller, D.}
\end{table}

A model based on Newton's law of cooling was proposed as
$$
f(x, \theta )=60+70 e^{ - \theta x }
$$
where $f$ is predicted temperature and $x$ is time.

\section{Puromycin}

Data on the ``velocity''
of an enzymatic reaction were obtained by \citeasnoun{trel:1974}.
%\glossary{ Treloar, M.A.}
The number of counts per minute of radioactive product from the reaction
was measured
as a function of substrate concentration in parts per million (ppm) and
from these counts the initial rate, or ``velocity,''
of the reaction was
calculated (counts/min$^{2}$).
The experiment was conducted once with the enzyme treated with
Puromycin, [(a) in Table~\ref{atbl:mic}] and once with the enzyme
untreated (b).
\begin{table}
  \caption{\label{atbl:mic}
  Reaction velocity versus substrate concentration for the Puromycin
  experiment.}
  \begin{center}
    \begin{tabular}{r r r}\hline
      Substrate&Velocity\\
      Concentration&(counts/min$^{2}$)\\
      (ppm)&(a) Treated&(b) Untreated\\ \hline
      0.02&76&67\\
      &47&51\\
      0.06&97&84\\
      &107&86\\
      0.11&123&98\\
      &139&115\\
      0.22&159&131\\
      &152&124\\
      0.56&191&144\\
      &201&158\\
      1.10&207&160\\
      &200\\ \hline
    \end{tabular}
    \begin{quote}\small
      Copyright 1974 by M. A. Treloar.  Reproduced from ``Effects
      of Puromycin on Galactosyltransferase of Golgi Membranes,''
      Master's Thesis, University of Toronto.  Reprinted with permission
      of the author.
    \end{quote}
  \end{center}
\end{table}
The velocity is assumed to depend on the substrate concentration
according to the Michaelis--Menten equation.
It was hypothesized that the ultimate velocity parameter
($\theta_1$) should be affected by introduction of the Puromycin,
but not the half-velocity parameter ($\theta_2$).

The Michaelis--Menten model is
$$
f(x, \btheta ) ={\theta_1 x \over \theta_{2+x}}
$$
where $f$ is predicted velocity and $x$ is substrate concentration.

\section{BOD}

Data on biochemical oxygen demand (BOD) were obtained by \citeasnoun{mars:1967}.
%\glossary{ Marske, D.}
To determine the BOD, a sample of stream water was taken, injected
with soluble organic matter, inorganic nutrients, and dissolved
oxygen, and subdivided into BOD bottles.  Each bottle was innoculated
with a mixed culture of microorganisms, sealed, and incubated at
constant temperature, and then the bottles were opened periodically
and analyzed for dissolved oxygen concentration, from which the BOD
was calculated in milligrams per liter (mg/l).  (See Table~\ref{atbl:bod})  The
values shown are the averages of two analyses on each bottle.
\begin{table}
  \caption{\label{atbl:bod}
  Biochemical oxygen demand versus time.}
  \begin{center}
    \begin{tabular}{r r r r}\hline
      &\multicolumn{1}{c}{Biochemical}&&\multicolumn{1}{c}{Biochemical}\\
      &\multicolumn{1}{c}{Oxygen}&&\multicolumn{1}{c}{Oxygen}\\
      \multicolumn{1}{c}{Time}&\multicolumn{1}{c}{Demand}&
      \multicolumn{1}{c}{Time}&\multicolumn{1}{c}{Demand}\\
      \multicolumn{1}{c}{(days)}&\multicolumn{1}{c}{(mg/l)}&
      \multicolumn{1}{c}{(days)}&\multicolumn{1}{c}{(mg/l)}\\
      \hline
      1&8.3&4&16.0\\
      2&10.3&5&15.6\\
      3&19.0&7&19.8\\
    \end{tabular}
  \end{center}
  \begin{quote}\small
    Copyright 1967 by D. Marske.  Reproduced from ``Biochemical
    Oxygen Demand Data Interpretation Using Sum of Squares Surface,''
    M.Sc.~Thesis, University of Wisconsin--Madison.  Reprinted with
    permission of the author.
  \end{quote}
\end{table}

A model was derived based on exponential decay with a fixed rate
constant as
$$
f(x, \btheta ) =\theta_1
( 1 - e^{ \theta_2 x } )
$$
where $f$ is predicted biochemical oxygen demand and $x$ is time.
\section{Isomerization}

Data on the reaction rate of the catalytic isomerization of
$n$-pentane to isopentane versus the partial pressures of hydrogen,
$n$-pentane, and isopentane were given in \citeasnoun{carr:1960} and
are reproduced in Table~\ref{atbl:iso}.
%\glossary{ Carr, N.L.}
\begin{table}
  \caption{\label{atbl:iso}
  Reaction rate for isomerization of $n$-pentane to isopentane.}
  \begin{center}
    \begin{tabular}{r r r r}
      \hline
      \multicolumn{3}{c}{Partial Pressure (psia)}&
      \multicolumn{1}{c}{Reaction}\\
      &&&\multicolumn{1}{c}{Rate}\\
      \multicolumn{1}{c}{Hydrogen}&\multicolumn{1}{c}{$n$-Pentane}&
      \multicolumn{1}{c}{Isopentane}&\multicolumn{1}{c}{(hr$^{-1}$)}\\
      \hline
      205.8&90.9&37.1&3.541\\
      404.8&92.9&36.3&2.397\\
      209.7&174.9&49.4&6.694\\
      401.6&187.2&44.9&4.722\\
      224.9&92.7&116.3&0.593\\
      402.6&102.2&128.9&0.268\\
      212.7&186.9&134.4&2.797\\
      406.2&192.6&134.9&2.451\\
      133.3&140.8&87.6&3.196\\
      470.9&144.2&86.9&2.021\\
      300.0&68.3&81.7&0.896\\
      301.6&214.6&101.7&5.084\\
      297.3&142.2&10.5&5.686\\
      314.0&146.7&157.1&1.193\\
      305.7&142.0&86.0&2.648\\
      300.1&143.7&90.2&3.303\\
      305.4&141.1&87.4&3.054\\
      305.2&141.5&87.0&3.302\\
      300.1&83.0&66.4&1.271\\
      106.6&209.6&33.0&11.648\\
      417.2&83.9&32.9&2.002\\
      251.0&294.4&41.5&9.604\\
      250.3&148.0&14.7&7.754\\
      145.1&291.0&50.2&11.590\\
      \hline
    \end{tabular}
  \end{center}
\begin{quote}\small
  Copyright 1960 by the American Chemical Society.  Reprinted with
  permission from {\em Industrial and Engineering Chemistry}, {\bf 52},
  391--396.
\end{quote}
\end{table}
Isomerization is a chemical process in which a complex chemical is
converted into more simple units, called isomers: catalytic
isomerization employs catalysts to speed the reaction.
The reaction rate depends on various factors, such as partial
pressures of the products and the concentration of the catalyst.
The differential reaction rate was expressed as grams of isopentane
produced per gram of catalyst per hour (hr$^{-1}$), and the
instantaneous partial pressure of a component was calculated as
the mole fraction of the component times the total pressure, in pounds
per square inch absolute (psia).

A common form of model for the reaction rate is the Hougen--Watson model
\cite{houg:wats:1947},
%\glossary{ Hougen, O.A.}
%\glossary{ Watson, K.M.}
of which the following is a special case,
$$
f ( \bx , \btheta )  =
{\theta_1 \theta_3 ( x_2 - x_3 / 1.632 )  \over 1 + \theta_2 x_1 +
\theta_3 x_2 + \theta_4 x_3}
$$
where $f$ is predicted reaction rate, $x_{1}$ is partial pressure of
hydrogen, $x_{2}$ is partial pressure of isopentane, and $x_{3}$ is
partial pressure of $n$-pentane.

\section{$\alpha$-Pinene}

Data on the thermal isomerization of
$\alpha$-pinene, a component of turpentine,
were reported in \citeasnoun{fugu:hawk:1947}.
%\glossary{ Fuguitt, R.E.}
%\glossary{ Hawkins, J.E.}
In this experiment, the relative concentrations (\%) of
$\alpha$-pinene
and three by-products were measured at each of eight times,
and the relative concentration of a fourth by-product was imputed
from the other concentrations.
(See Table~\ref{atbl:bhme})
\begin{table}
  \caption{\label{atbl:bhme}
  Relative concentrations of products versus time for thermal
  isomerization of $\alpha$-pinene at 189.5$^\circ$C.}
  \begin{center}
    \begin{tabular}{r r r r r r}\hline
      \multicolumn{1}{c}{Time}&\multicolumn{1}{c}{$\alpha$-Pinene}&
      \multicolumn{1}{c}{Dipentene}&\multicolumn{1}{c}{Alloocimene}&
      \multicolumn{1}{c}{Pyronene}&\multicolumn{1}{c}{Dimer}\\
      \multicolumn{1}{c}{(min)}&\multicolumn{1}{c}{(\%)}&\multicolumn{1}{c}{(\%)}&
      \multicolumn{1}{c}{(\%)}&\multicolumn{1}{c}{(\%)}&\multicolumn{1}{c}{(\%)}\\
      \hline
      1230&88.35&7.3&2.3&0.4&1.75\\
      3060&76.4&15.6&4.5&0.7&2.8\\
      4920&65.1&23.1&5.3&1.1&5.8\\
      7800&50.4&32.9&6.0&1.5&9.3\\
      10680&37.5&42.7&6.0&1.9&12.0\\
      15030&25.9&49.1&5.9&2.2&17.0\\
      22620&14.0&57.4&5.1&2.6&21.0\\
      36420&4.5&63.1&3.8&2.9&25.7\\
      \hline
    \end{tabular}
  \end{center}
\begin{quote}\small
  Copyright 1947 by the American Chemical Society.  Reprinted with
  permission from {\em Journal of the American Chemical Society},
  {\bf 69}, 319--322.
\end{quote}
\end{table}
The initial concentration of $\alpha$-pinene was 100\%.

A linear kinetic model, shown in Figure~\ref{fig:pinene},
\expandafter\ifx\csname graph\endcsname\relax \csname newbox\endcsname\graph\fi
\expandafter\ifx\csname graphtemp\endcsname\relax \csname newdimen\endcsname\graphtemp\fi
\setbox\graph=\vtop{\vskip 0pt\hbox{%
    \special{pn 8}%
    \special{ar 250 250 250 250 0 6.28319}%
    \graphtemp=.5ex\advance\graphtemp by 0.250in
    \rlap{\kern 0.250in\lower\graphtemp\hbox to 0pt{\hss $f_{1}$\hss}}%
    \special{ar 1250 250 250 250 0 6.28319}%
    \graphtemp=.5ex\advance\graphtemp by 0.250in
    \rlap{\kern 1.250in\lower\graphtemp\hbox to 0pt{\hss $f_{2}$\hss}}%
    \special{ar 250 1000 250 250 0 6.28319}%
    \graphtemp=.5ex\advance\graphtemp by 1.000in
    \rlap{\kern 0.250in\lower\graphtemp\hbox to 0pt{\hss $f_{3}$\hss}}%
    \special{ar 1250 1000 250 250 0 6.28319}%
    \graphtemp=.5ex\advance\graphtemp by 1.000in
    \rlap{\kern 1.250in\lower\graphtemp\hbox to 0pt{\hss $f_{4}$\hss}}%
    \special{ar 250 1750 250 250 0 6.28319}%
    \graphtemp=.5ex\advance\graphtemp by 1.750in
    \rlap{\kern 0.250in\lower\graphtemp\hbox to 0pt{\hss $f_{5}$\hss}}%
    \special{pa 500 250}%
    \special{pa 1000 250}%
    \special{fp}%
    \special{sh 1.000}%
    \special{pa 900 225}%
    \special{pa 1000 250}%
    \special{pa 900 275}%
    \special{pa 900 225}%
    \special{fp}%
    \graphtemp=\baselineskip\multiply\graphtemp by -1\divide\graphtemp by 2
    \advance\graphtemp by .5ex\advance\graphtemp by 0.250in
    \rlap{\kern 0.750in\lower\graphtemp\hbox to 0pt{\hss $\theta_{1}$\hss}}%
    \special{pa 250 500}%
    \special{pa 250 750}%
    \special{fp}%
    \special{sh 1.000}%
    \special{pa 275 650}%
    \special{pa 250 750}%
    \special{pa 225 650}%
    \special{pa 275 650}%
    \special{fp}%
    \graphtemp=.5ex\advance\graphtemp by 0.625in
    \rlap{\kern 0.250in\lower\graphtemp\hbox to 0pt{\hss $\theta_{2}$ }}%
    \special{pa 500 1000}%
    \special{pa 1000 1000}%
    \special{fp}%
    \special{sh 1.000}%
    \special{pa 900 975}%
    \special{pa 1000 1000}%
    \special{pa 900 1025}%
    \special{pa 900 975}%
    \special{fp}%
    \graphtemp=\baselineskip\multiply\graphtemp by -1\divide\graphtemp by 2
    \advance\graphtemp by .5ex\advance\graphtemp by 1.000in
    \rlap{\kern 0.750in\lower\graphtemp\hbox to 0pt{\hss $\theta_{3}$\hss}}%
    \special{pa 73 1177}%
    \special{pa 73 1573}%
    \special{fp}%
    \special{sh 1.000}%
    \special{pa 98 1473}%
    \special{pa 73 1573}%
    \special{pa 48 1473}%
    \special{pa 98 1473}%
    \special{fp}%
    \graphtemp=.5ex\advance\graphtemp by 1.375in
    \rlap{\kern 0.073in\lower\graphtemp\hbox to 0pt{\hss $\theta_{4}$ }}%
    \special{pa 427 1177}%
    \special{pa 427 1573}%
    \special{fp}%
    \special{sh 1.000}%
    \special{pa 402 1277}%
    \special{pa 427 1177}%
    \special{pa 452 1277}%
    \special{pa 402 1277}%
    \special{fp}%
    \graphtemp=.5ex\advance\graphtemp by 1.375in
    \rlap{\kern 0.427in\lower\graphtemp\hbox to 0pt{\hss $\theta_{5}$ }}%
    \hbox{\vrule depth2.000in width0pt height 0pt}%
    \kern 1.500in
  }%
}%

\begin{figure}
  \centerline{\box\graph}
  \caption{\label{fig:pinene}
  System diagram for $\alpha$-pinene model where $f_{1}$ is
  $\alpha$-pinene concentration, $f_{2}$ is dipentene concentration,
  $f_{3}$ is alloocimene concentration, $f_{4}$ is pyronene
  concentration, and $f_{5}$ is dimer concentration.}
\end{figure}
was proposed in \citeasnoun{box:hunt:macg:erja:1973}.
%\glossary{ Box, G.E.P.}
%\glossary{ Hunter, W.G.}
%\glossary{ MacGregor, J.M.}
%\glossary{ Erjavec, J.}
This model provides for the production of dipentene and alloocimene,
which in turn yields $\alpha$- and $\beta$-pyronene and a dimer.

\section{Sulfisoxazole}

Data on the metabolism of sulfisoxazole were obtained by
\citeasnoun{kapl:wein:abru:lewi:1972} and are reproduced in
%\glossary{ Kaplan, S.A.}
%\glossary{ Weinfeld, R.E.}
%\glossary{ Abruzzo, C.W.}
%\glossary{ Lewis, M.}
Table~\ref{atbl:sulf}.
\begin{table}
  \caption{\label{atbl:sulf}
  Sulfisoxazole concentration versus time.}
  \begin{center}
    \begin{tabular}{r r r r}\hline
      \multicolumn{1}{c}{Time}&\multicolumn{1}{c}{Sulfisoxazole}&
      \multicolumn{1}{c}{Time}&\multicolumn{1}{c}{Sulfisoxazole}\\
      &\multicolumn{1}{c}{Conc.}&&\multicolumn{1}{c}{Conc.}\\
      \multicolumn{1}{c}{(min)}&\multicolumn{1}{c}{($\mu$g/ml)}&
      \multicolumn{1}{c}{(min)}&\multicolumn{1}{c}{($\mu$g/ml)}\\
      \hline
      0.25&215.6&3.00&101.2\\
      0.50&189.2&4.00&88.0\\
      0.75&176.0&6.00&61.6\\
      1.00&162.8&12.00&22.0\\
      1.50&138.6&24.00&4.4\\
      2.00&121.0&48.00&0.1\\
      \hline
    \end{tabular}
  \end{center}
\begin{quote}\small
  Reproduced from the {\em Journal of the American Pharmaceutical
  Association}, 1972, {\bf 61}, 773--778, with permission of the copyright
  owner, the American Pharmaceutical Association.
\end{quote}
\end{table}
In this experiment, sulfisoxazole was administered
to a subject intravenously, blood samples were taken at specified times,
and the concentration of sulfisoxazole in the plasma in micrograms per
milliliter ($\mu$g/ml) was measured.

For the intravenous data, a 2-compartment model was proposed, which we
write as a sum of two exponentials,
$$
f(x,\btheta)=\theta_1 e^{-\theta_2 x}+\theta_3 e^{-\theta_4 x}
$$
where $f$ is predicted sulfisoxazole concentration and $x$ is time.

\section{Lubricant}

Data on the kinematic viscosity of a lubricant, in stokes, as a
function of temperature ($^\circ$C), and pressure in atmospheres (atm),
were obtained (see Table~\ref{atbl:lub})
\begin{table}
  \caption{\label{atbl:lub}
  Logarithm of lubricant viscosity versus pressure and temperature.}
  \begin{center}
    \begin{tabular}{r r r r}\hline
      \multicolumn{1}{c}{T = 0$^\circ$C}&\multicolumn{1}{c}{T = 25$^\circ$C}\\
      \multicolumn{1}{c}{Pressure}&\multicolumn{1}{c}{ln[viscosity}&
      \multicolumn{1}{c}{Pressure}&\multicolumn{1}{c}{ln[viscosity}\\
      \multicolumn{1}{c}{(atm)}&\multicolumn{1}{c}{(s)]}&
      \multicolumn{1}{c}{(atm)}&\multicolumn{1}{c}{(s)]}\\
      \hline
      1.000&5.10595&1.000&4.54223\\
      740.803&6.38705&805.500&5.82452\\
      1407.470&7.38511&1505.920&6.70515\\
      363.166&5.79057&2339.960&7.71659\\
      1.000&5.10716&422.941&5.29782\\
      805.500&6.36113&1168.370&6.22654\\
      1868.090&7.97329&2237.290&7.57338\\
      3285.100&10.47250&4216.890&10.3540\\
      3907.470&11.92720&5064.290&11.9844\\
      4125.470&12.42620&5280.880&12.4435\\
      2572.030&9.15630&3647.270&9.52333\\
      &&2813.940&8.34496\\
      \hline
      \multicolumn{1}{c}{T = 37.8$^\circ$C}&\multicolumn{1}{c}{T = 98.9$^\circ$C}\\
      \multicolumn{1}{c}{Pressure}&\multicolumn{1}{c}{ln[viscosity}&
      \multicolumn{1}{c}{Pressure}&\multicolumn{1}{c}{ln[viscosity}\\
      \multicolumn{1}{c}{(atm)}&\multicolumn{1}{c}{(s)]}&
      \multicolumn{1}{c}{(atm)}&\multicolumn{1}{c}{(s)]}\\
      \hline
      516.822&5.17275&1.000&3.38099\\
      1737.990&6.64963&685.950&4.45783\\
      1008.730&5.80754&1423.640&5.20675\\
      2749.240&7.74101&2791.430&6.29101\\
      1375.820&6.23206&4213.370&7.32719\\
      191.084&4.66060&2103.670&5.76988\\
      1.000&4.29865&402.195&4.08766\\
      2922.940&7.96731&1.000&3.37417\\
      4044.600&9.34225&2219.700&5.83919\\
      4849.800&10.51090&3534.750&6.72635\\
      5605.780&11.82150&4937.710&7.76883\\
      6273.850&13.06800&6344.170&8.91362\\
      3636.720&8.80445&7469.350&9.98334\\
      1948.960&6.85530&5640.940&8.32329\\
      1298.470&6.11898&4107.890&7.13210\\
      \hline
    \end{tabular}
  \end{center}
\begin{quote}\small
  Reprinted with permission of H. N. Linssen.
\end{quote}
\end{table}
and an empirical model was proposed for the logarithm of
the viscosity, as discussed in \citeasnoun{lins:1975}.
%\glossary{ Linssen, H.N.}

The proposed model is
$$
f ( \bx , \btheta ) = {\theta_1 \over  \theta_2 + x_1 } +
\theta_3 x_2 + \theta_4 x_2^2 +
\theta_5 x_2^3 +
( \theta_6 + \theta_7 x_2^2 )  x_2 
\exp \left( { - x_1   \over  \theta_8 + \theta_9 x_2^2 }
\right)
$$
where $f$ is predicted ln(viscosity), $x_{1}$ is temperature, and
$x_{2}$ is pressure.

\section{Chloride}

Data on the rate of transport of sulfite ions from blood cells
suspended in a salt solution were obtained by W. H. Dennis and
P. Wood at the University of Wisconsin, and analyzed by \citeasnoun{sred:1970}.
%\glossary{ Sredni, J.}
The chloride concentration (\%) was determined from a continuous curve
generated from electrical potentials.
(See Table~\ref{atbl:chlor})
\begin{table}
  \caption{\label{atbl:chlor}
  Chloride ion concentration versus time.}
  \begin{center}
    \begin{tabular}{r r r r r r}\hline
      \multicolumn{1}{c}{Time}&\multicolumn{1}{c}{Conc.}&
      \multicolumn{1}{c}{Time}&\multicolumn{1}{c}{Conc.}&
      \multicolumn{1}{c}{Time}&\multicolumn{1}{c}{Conc.}\\
      \multicolumn{1}{c}{(min)}&\multicolumn{1}{c}{(\%)}&
      \multicolumn{1}{c}{(min)}&\multicolumn{1}{c}{(\%)}&
      \multicolumn{1}{c}{(min)}&\multicolumn{1}{c}{(\%)}\\
      \hline
      2.45&17.3&4.25&22.6&6.05&26.6\\
      2.55&17.6&4.35&22.8&6.15&27.0\\
      2.65&17.9&4.45&23.0&6.25&27.0\\
      2.75&18.3&4.55&23.2&6.35&27.0\\
      2.85&18.5&4.65&23.4&6.45&27.0\\
      2.95&18.9&4.75&23.7&6.55&27.3\\
      3.05&19.0&4.85&24.0&6.65&27.8\\
      3.15&19.3&4.95&24.2&6.75&28.1\\
      3.25&19.8&5.05&24.5&6.85&28.1\\
      3.35&19.9&5.15&25.0&6.95&28.1\\
      3.45&20.2&5.25&25.4&7.05&28.4\\
      3.55&20.5&5.35&25.5&7.15&28.6\\
      3.65&20.6&5.45&25.9&7.25&29.0\\
      3.75&21.1&5.55&25.9&7.35&29.2\\
      3.85&21.5&5.65&26.3&7.45&29.3\\
      3.95&21.9&5.75&26.2&7.55&29.4\\
      4.05&22.0&5.85&26.5&7.65&29.4\\
      4.15&22.3&5.95&26.5&7.75&29.4\\
      \hline
    \end{tabular}
  \end{center}
\begin{quote}\small
  Reproduced from J. Sredni, ``Problems of Design, Estimation, and Lack
  of Fit in Model Building,'' Ph.D.~Thesis, University of
  Wisconsin--Madison, 1970, with permission of the author.
\end{quote}
\end{table}

A model was derived from the theory of ion transport as
$$
f ( x , \btheta ) = \theta_1
( 1 - \theta_2 e^{ - \theta_3 x } )
$$
where $f$ is predicted chloride concentration and $x$ is time.
\section{Ethyl Acrylate}

Data on the metabolism of ethyl acrylate
were obtained by giving rats a bolus of
radioactively tagged ethyl acrylate \cite{watt:debe:stir:1986}.
%\glossary{ Stiratelli, R.G.}
%\glossary{ deBethizy, D.}
%\glossary{ Watts, D.G.}
Each rat was given a measured dose of the compound via stomach
intubation and placed in an enclosed cage from which the air
could be drawn through a bubble chamber.
The exhalate was bubbled through the chamber, and at a
specified time the bubble chamber was replaced by a fresh one, so
that the measured response was the accumulated CO$_{2}$ during
the collection interval.
The response reported in Table~\ref{atbl:co2} is the average, for nine rats,
\begin{table}
  \caption{\label{atbl:co2}
  Collection intervals and averages of normalized exhaled CO$_2$.}
  \begin{center}
    \begin{tabular}{r r r}\hline
      \multicolumn{1}{c}{Collection}&\\
      \multicolumn{1}{c}{Interval (hr)}&\multicolumn{1}{c}{CO$_{2}$}\\
      \multicolumn{1}{c}{Start}&\multicolumn{1}{c}{Length}&
      \multicolumn{1}{c}{(g)}\\
      \hline
      0.0&0.25&0.01563\\
      0.25&0.25&0.04190\\
      0.5&0.25&0.05328\\
      0.75&0.25&0.05226\\
      1.0&0.5&0.08850\\
      1.5&0.5&0.06340\\
      2.0&2.0&0.13419\\
      4.0&2.0&0.04502\\
      6.0&2.0&0.02942\\
      8.0&16.0&0.02716\\
      24.0&24.0&0.01037\\
      48.0&24.0&0.00602\\
      \hline
    \end{tabular}
  \end{center}
\begin{quote}\small
  Reproduced with permission.
\end{quote}
\end{table}
of the amount of accumulated
CO$_{2}$ normalized by actual dose,
in units of grams CO$_{2}$ per gram acrylate per gram rat.
An empirical model with three exponential terms was
determined from inspection of plots of the data and physical reasoning.
Logarithms of the integrated function were fitted to logarithms of the
data, using the refinements of Section 3.9.

The integrated model is written
\begin{eqnarray*}
  F(\bx,\btheta)&=&-{\theta_4+\theta_5\over\theta_1}e^{-\theta_1 x_1}
  (1-e^{-\theta_1 x_2})\\
  &&+{\theta_4\over\theta_2}e^{-\theta_2 x_1}(1-e^{-\theta_2 x_2})
  +{\theta_5\over\theta_3}e^{-\theta_3 x_1}(1-e^{-\theta_3 x_2})
\end{eqnarray*}
where $F$ is predicted CO$_{2}$ exhaled during an interval,
$x_{1}$ is interval starting time, and $x_{2}$ is interval
duration.

\section{Saccharin}

Data on the metabolism of saccharin compounds were obtained by \citeasnoun{renw:1982}.
%\glossary{ Renwick, A.G.}
In this experiment, a rat received a single bolus of saccharin, and the
amount of saccharin excreted was measured by collecting urine in
contiguous time intervals.
The measured response was the level of radioactivity of the urine, which
was converted to amount of saccharin in micrograms ($\mu$g).
(See Table~\ref{atbl:sac}.)
\begin{table}
  \caption{\label{atbl:sac}
  Collection intervals and excreted saccharin amounts.}
  \begin{center}
    \begin{tabular}{r r r}
      \hline
      \multicolumn{1}{c}{Collection}&\\
      \multicolumn{1}{c}{Interval (min)}&\multicolumn{1}{c}{Saccharin}\\
      \multicolumn{1}{c}{Start}&\multicolumn{1}{c}{Length}&
      \multicolumn{1}{c}{($\mu$g)}\\
      \hline
      0&5&7518\\
      5&10&6275\\
      15&15&4989\\
      30&15&2580\\
      45&15&1485\\
      60&15&861\\
      75&15&561\\
      90&15&363\\
      105&15&300\\
    \end{tabular}
  \end{center}
\begin{quote}\small
  From ``Pharmacokinetics in Toxicology,'' by A. G. Renwick, in {\em
  Principles and Methods of Toxicology}, A. Wallace Hayes, Ed., Raven
  Press, 1982.  Reprinted with permission of the publisher.
\end{quote}
\end{table}
An empirical compartment model with two exponential terms was
determined from inspection of plots of the data.
Logarithms of the integrated function were fitted to logarithms of the
data, using the refinements of Section 3.9.

The integrated model is written
$$
F ( \bx , \btheta ) = 
 { \theta_3   \over  \theta_1 }
e^{ - \theta_1 x_1 }
( 1 - e^{ - \theta_1 x_2} )
 + { \theta_4   \over  \theta_2}
e^{ - \theta_2 x_1 }
( 1 - e^{ - \theta_2 x_2} )
$$
where $F$ is predicted saccharin excreted during an interval,
$x_{1}$ is interval starting time, and $x_{2}$ is interval
duration.

\section{Nitrite Utilization}

Data on the utilization of nitrite in bush beans as a function of
light intensity were obtained by J.R. Elliott and D.R. Peirson of
Wilfrid Laurier University.
%\glossary{ Elliott, J.R.}
%\glossary{ Peirson, D.R.}
Portions of primary leaves from three 16-day-old bean plants were
subjected to eight levels of light intensity measured in microeinsteins
per square metre per second ($\mu {\rm E/m}^{2}$s) and the nitrite
utilization in nanomoles of NO$_2^{-}$ per gram per hour
(nmol/ghr) was measured.  The experiment was repeated on a different
day.  (See Table~\ref{atbl:nitrite})
\begin{table}
  \caption{\label{atbl:nitrite}
  Nitrite utilization versus light intensity.}
  \begin{center}
    \begin{tabular}{r r r}
      \hline
      \multicolumn{1}{c}{Light}&\multicolumn{1}{c}{Nitrite Utilization
      Intensity}&
      \multicolumn{1}{c}{(nmol/ghr)}\\
      \multicolumn{1}{c}{($\mu {\rm E/m}^{2}$s)}&
      \multicolumn{1}{c}{Day 1}&\multicolumn{1}{c}{Day 2}\\
      \hline
      2.2&256&549\\
      &685&1550\\
      &1537&1882\\
      5.5&2148&1888\\
      &2583&3372\\
      &3376&2362\\
      9.6&3634&4561\\
      &4960&4939\\
      &3814&4356\\
      17.5&6986&7548\\
      &6903&7471\\
      &7636&7642\\
      27.0&9884&9684\\
      &11597&8988\\
      &10221&8385\\
      46.0&17319&13505\\
      &16539&15324\\
      &15047&15430\\
      94.0&19250&17842\\
      &20282&18185\\
      &18357&17331\\
      170.0&19638&18202\\
      &19043&18315\\
      &17475&15605\\
    \end{tabular}
  \end{center}
\begin{quote}\small
  Reprinted with permission of J. R. Elliott and D. R. Peirson.
\end{quote}
\end{table}

An empirical model was suggested to satisfy the requirements of
zero nitrite utilization at zero light intensity and approach to an
asymptote as light intensity increased.
Two models were fitted which rose to a peak and then began to decline,
as described in Section 3.12.
These models are
$$
f(x, \btheta ) = { \theta_1 x   \over  \theta_2 + x + \theta_3  x^2 }
$$
and
$$
f(x, \btheta ) = \theta_1 ( e^{ - \theta_3 x } -
e^{ - \theta_2 x })
$$
where $f$ is predicted nitrite utilization and $x$ is light intensity.

\section{s-PMMA}

Data on the dielectric behavior of syndiotactic
poly(methylmethacrylate) (s-PMMA) were obtained
by \citeasnoun{havr:nega:1967}.
%\glossary{ Havriliak, S.Jr.}
%\glossary{ Negami, S.}
A disk of the polymer was inserted between the two
metal electrodes of a dielectric cell which formed one arm of a
four-armed electrical bridge.
The bridge was powered by an oscillating voltage whose frequency
$f$ could be changed from 5 to
500000 hertz (Hz), and bridge balance was achieved using capacitance
and conductance standards.
The complex dielectric constant was calculated using changes
from the standards relative to the cell dielectric constant.
Measurements were made by simultaneously adjusting the capacitance
(real) and the conductance (imaginary) arms of the bridge
when it was excited at a specific frequency.
The measured responses were the relative capacitance and relative
conductance (dimensionless).
(See Table~\ref{atbl:spmma})
\begin{table}
  \caption{\label{atbl:spmma}
  Real and imaginary dielectric constant versus frequency
  for s-PMMA at 86.7$^\circ$F.}
  \begin{center}
    \begin{tabular}{r r r r r r}
      \hline
      &\multicolumn{1}{c}{Relative}&&\multicolumn{1}{c}{Relative}\\
      \multicolumn{1}{c}{Frequency}&\multicolumn{1}{c}{Impedance}&
      \multicolumn{1}{c}{Frequency}&\multicolumn{1}{c}{Impedance}\\
      \multicolumn{1}{c}{(Hz)}&\multicolumn{1}{c}{Real}&
      \multicolumn{1}{c}{Imag}&\multicolumn{1}{c}{(Hz)}&
      \multicolumn{1}{c}{Real}&\multicolumn{1}{c}{Imag}\\
      \hline
      30&4.220&0.136&3000&3.358&0.305\\
      50&4.167&0.167&5000&3.258&0.289\\
      70&4.132&0.188&7000&3.193&0.277\\
      100&4.038&0.212&10000&3.128&0.255\\
      150&4.019&0.236&15000&3.059&0.240\\
      200&3.956&0.257&20000&2.984&0.218\\
      300&3.884&0.276&30000&2.934&0.202\\
      500&3.784&0.297&50000&2.876&0.182\\
      700&3.713&0.309&70000&2.838&0.168\\
      1000&3.633&0.311&100000&2.798&0.153\\
      1500&3.540&0.314&150000&2.759&0.139\\
      2000&3.433&0.311\\
      \hline
    \end{tabular}
  \end{center}
  \begin{quote}\small
    From ``Analytic Representation of Dielectric Constants: A
    Complex Multiresponse Problem,'' by S. Havriliak, Jr. and D. G.
    Watts, in {\em Design, Data, and Analysis} , Colin L. Mallows, Ed.,
    Wiley, 1987.  Reprinted with permission of the publisher.
  \end{quote}
\end{table}

The model is an empirical generalization of two models based on
theory.
It is written
$$
f ( x , \btheta ) =\theta_2+
{\theta_1-\theta_2  \over  \left[ 1+\left( i 2 \pi xe^{{-} \theta_3} \right)^{{\theta}_4}
\right]^{{\theta}_5} }
$$
where $f$ is predicted relative complex impedance and $x$ is
frequency.

\section{Tetracycline}

Data on the metabolism of tetracycline were presented in \citeasnoun{wagn:1967}.
%\glossary{ Wagner, J.G.}
In this experiment, a tetracycline compound was administered
orally to a subject and the concentration of tetracycline
hydrochloride in the serum in micrograms per milliliter ($\mu$g/ml)
was measured over a period of 16 hours.
(See Table~\ref{atbl:tet})
\begin{table}
  \caption{\label{atbl:tet}
  Tetracycline concentration versus time.}
  \begin{center}
    \begin{tabular}{r r r r}
      \hline
      &\multicolumn{1}{c}{Tetracycline}&&\multicolumn{1}{c}{Tetracycline}\\
      \multicolumn{1}{c}{Time}&\multicolumn{1}{c}{Conc.}&
      \multicolumn{1}{c}{Time}&\multicolumn{1}{c}{Conc.}\\
      \multicolumn{1}{c}{(hr)}&\multicolumn{1}{c}{($\mu$g/ml)}&
      \multicolumn{1}{c}{(hr)}&\multicolumn{1}{c}{($\mu$g/ml)}\\
      \hline
      1&0.7&8&0.8 \\
      2&1.2&10&0.6\\
      3&1.4&12&0.5\\
      4&1.4&16&0.3\\
      6&1.1\\
      \hline
    \end{tabular}
  \end{center}
\begin{quote}\small
  From ``Use of Computers in Pharmacokinetics,'' by J.G. Wagner, in {\em
  Journal of Clinical Pharmacology and Therapeutics}, 1967, {\bf 8}, 201.
  Reprinted with permission of the publisher.
\end{quote}
\end{table}

A 2-compartment model was proposed, and dead time was incorporated
as
$$
f ( x , \btheta ) =\theta_3
[ e^{{-} \theta_1 ( x - \theta_4 )}-
  e^{{-} \theta_2 ( x - \theta_4 )}]
$$
where $f$ is predicted tetracycline hydrochloride concentration and
$x$ is time.

\section{Oil Shale}

Data on the pyrolysis of oil shale were obtained by
\citeasnoun{hubb:robi:1950} and are reproduced in Table~\ref{atbl:oil}.
%\glossary{ Hubbard, A.B.}
%\glossary{ Robinson, W.E.}
\begin{table}
  \caption{\label{atbl:oil}
  Relative concentration of bitumen and oil versus time and
  temperature for pyrolysis of oil shale.}
  \begin{center}
    \begin{tabular}{r r r r r r}
      \hline
      \multicolumn{1}{c}{T = 673K}&\multicolumn{1}{c}{T = 698K}\\
      \multicolumn{1}{c}{Time}&\multicolumn{1}{c}{Concentration (\%)}&
      \multicolumn{1}{c}{Time}&\multicolumn{1}{c}{Concentration
      (\%)}\\
      \multicolumn{1}{c}{(min)}&\multicolumn{1}{c}{Bitumen}&
      \multicolumn{1}{c}{Oil}&\multicolumn{1}{c}{(min)}&
      \multicolumn{1}{c}{Bitumen}&\multicolumn{1}{c}{Oil}\\
      \hline
      5&0.0&0.0&5.0&6.5&0.0\\
      7&2.2&0.0&7.0&14.4&1.4\\
      10&11.5&0.7&10.0&18.0&10.8\\
      15&13.7&7.2&12.5&16.5&14.4\\
      20&15.1&11.5&15.0&29.5&21.6\\
      25&17.3&15.8&17.5&23.7&30.2\\
      30&17.3&20.9&20.0&36.7&33.1\\
      40&20.1&26.6&25.0&27.3&40.3\\
      50&20.1&32.4&30.0&16.5&47.5\\
      60&22.3&38.1&40.0&7.2&55.4\\
      80&20.9&43.2&50.0&3.6&56.8\\
      100&11.5&49.6&60.0&2.2&59.7\\
      120&6.5&51.8&\\
      150&3.6&54.7&\\
      \hline
      \multicolumn{1}{c}{T = 723K}&\multicolumn{1}{c}{T = 748K}\\
      \multicolumn{1}{c}{Time}&\multicolumn{1}{c}{Concentration (\%)}&
      \multicolumn{1}{c}{Time}&\multicolumn{1}{c}{Concentration (\%)}\\
      \multicolumn{1}{c}{(min)}&\multicolumn{1}{c}{Bitumen}&
      \multicolumn{1}{c}{Oil}&\multicolumn{1}{c}{(min)}&
      \multicolumn{1}{c}{Bitumen}&\multicolumn{1}{c}{Oil}\\
      \hline
      5.0&8.6&0.0&3.0&0.7&0.0\\
      7.5&15.8&2.9&4.5&17.3&2.9\\
      8.0&25.9&16.5&5.0&23.0&17.3\\
      9.0&25.2&24.4&5.5&24.4&20.9\\
      10.0&26.6&29.5&6.0&23.0&25.9\\
      11.0&33.8&35.2&6.5&33.1&29.5\\
      12.5&25.9&39.5&7.0&31.6&33.8\\
      15.0&20.1&45.3&8.0&20.9&45.3\\
      17.5&12.9&43.1&9.0&10.1&53.2\\
      17.5&9.3&54.6&10.0&4.3&58.2\\
      20.0&3.6&59.7&12.5&0.7&57.5\\
      20.0&2.2&53.9&15.0&0.7&61.1\\
      \hline
      \multicolumn{1}{c}{T = 773K}&\multicolumn{1}{c}{T = 798K}\\
      \multicolumn{1}{c}{Time}&\multicolumn{1}{c}{Concentration (\%)}&
      \multicolumn{1}{c}{Time}&\multicolumn{1}{c}{Concentration (\%)}\\
      \multicolumn{1}{c}{(min)}&\multicolumn{1}{c}{Bitumen}&
      \multicolumn{1}{c}{Oil}&\multicolumn{1}{c}{(min)}&
      \multicolumn{1}{c}{Bitumen}&\multicolumn{1}{c}{Oil}\\
      \hline
      3.0&6.5&0.0&3.00&25.2&20.9\\
      4.0&24.4&23.0&3.25&33.1&25.2\\
      4.5&26.6&32.4&3.50&21.6&17.3\\
      5.0&25.9&37.4&4.00&20.9&36.7\\
      5.5&17.3&45.3&5.00&4.3&56.8\\
      6.0&21.6&45.3&7.00&0.0&61.8\\
      6.5&1.4&57.5&\\
      10.0&0.0&60.4&\\
    \end{tabular}
  \end{center}
\begin{quote}\small
  From ``A Thermal Decomposition Study of Colorado Oil Shale,'' Hubbard,
  A.B. and Robinson, W.E., U.S. Bureau of Mines, Rept. Invest.  No. 4744,
  1950.
\end{quote}
\end{table}
Oil shale contains organic matter which is organically bonded to
the structure of the rock:
to extract oil from the rock, heat is applied, and so the technique
is called pyrolysis.
During pyrolysis, the benzene organic material, called kerogen,
decomposes chemically to oil and bitumen, and there are
unmeasured by-products of insoluble organic residues and light
gases.
The responses measured were the concentrations of oil and
bitumen (\%).
The initial concentration of kerogen was 100\%.
\citeasnoun{zieg:gorm:1980} proposed a linear kinetic model with
%\glossary{ Ziegel, E.R.}
%\glossary{ Gorman, J.W.}
the system diagram in Figure~\ref{fig:oil}.
\input{pic/oil}
\begin{figure}
  \centerline{\box\graph}
  \caption{\label{fig:oil}
  System diagram for oil shale model where $f_{1}$ is kerogen, $f_{2}$
  is bitumen, and $f_{3}$ is oil.}
\end{figure}

\section{Lipoproteins}

Data on lipoprotein metabolism were reported in
\citeasnoun{ande:1983}.
%\glossary{ Anderson, D.H.}
The response was the concentration, in percent, of a tracer in
the serum of a baboon given a bolus injection.
Measurements were made at half-day and day intervals.
(See Table~\ref{atbl:lipo})
\begin{table}
  \caption{\label{atbl:lipo}
  Lipoprotein tracer concentration versus time.}
  \begin{center}
    \begin{tabular}{r r r r}
      \hline
      &\multicolumn{1}{c}{Tracer}&&\multicolumn{1}{c}{Tracer}\\
      \multicolumn{1}{c}{Time}&\multicolumn{1}{c}{Conc.}&
      \multicolumn{1}{c}{Time}&\multicolumn{1}{c}{Conc.}\\
      \multicolumn{1}{c}{(days)}&\multicolumn{1}{c}{(\%)}&
      \multicolumn{1}{c}{(days)}&\multicolumn{1}{c}{(\%)}\\
      \hline
      0.5&46.10&5.0&3.19 \\
      1.0&25.90&6.0&2.40 \\
      1.5&17.00&7.0&1.82 \\
      2.0&12.10&8.0&1.41 \\
      3.0&7.22&9.0&1.00 \\
      4.0&4.51&10.0&0.94\\
      \hline
    \end{tabular}
  \end{center}
\begin{quote}\small
  From {\em Compartmental Modeling and Tracer Kinetics}, D. H.
  Anderson, p 211, 1983, Springer--Verlag.  Reproduced with permission of
  the author and the publisher.
\end{quote}
\end{table}
An empirical compartment model with two exponential terms was proposed,
based on inspection of plots of the data.
The system diagram of the final 3-compartment catenary
model fitted in Section 5.4 is given in
Figure~\ref{fig:tetra}.
\expandafter\ifx\csname graph\endcsname\relax \csname newbox\endcsname\graph\fi
\expandafter\ifx\csname graphtemp\endcsname\relax \csname newdimen\endcsname\graphtemp\fi
\setbox\graph=\vtop{\vskip 0pt\hbox{%
    \special{pn 8}%
    \special{ar 250 250 250 250 0 6.28319}%
    \graphtemp=.5ex\advance\graphtemp by 0.250in
    \rlap{\kern 0.250in\lower\graphtemp\hbox to 0pt{\hss $f_{1}$\hss}}%
    \special{ar 1250 250 250 250 0 6.28319}%
    \graphtemp=.5ex\advance\graphtemp by 0.250in
    \rlap{\kern 1.250in\lower\graphtemp\hbox to 0pt{\hss $f_{2}$\hss}}%
    \special{ar 2250 250 250 250 0 6.28319}%
    \graphtemp=.5ex\advance\graphtemp by 0.250in
    \rlap{\kern 2.250in\lower\graphtemp\hbox to 0pt{\hss $f_{3}$\hss}}%
    \special{pa 250 500}%
    \special{pa 250 1000}%
    \special{fp}%
    \special{sh 1.000}%
    \special{pa 275 900}%
    \special{pa 250 1000}%
    \special{pa 225 900}%
    \special{pa 275 900}%
    \special{fp}%
    \graphtemp=.5ex\advance\graphtemp by 0.750in
    \rlap{\kern 0.250in\lower\graphtemp\hbox to 0pt{\hss $\theta_{1}$ }}%
    \special{pa 427 73}%
    \special{pa 1073 73}%
    \special{fp}%
    \special{sh 1.000}%
    \special{pa 973 48}%
    \special{pa 1073 73}%
    \special{pa 973 98}%
    \special{pa 973 48}%
    \special{fp}%
    \graphtemp=\baselineskip\multiply\graphtemp by -1\divide\graphtemp by 2
    \advance\graphtemp by .5ex\advance\graphtemp by 0.073in
    \rlap{\kern 0.750in\lower\graphtemp\hbox to 0pt{\hss $\theta_{2}$\hss}}%
    \special{pa 427 427}%
    \special{pa 1073 427}%
    \special{fp}%
    \special{sh 1.000}%
    \special{pa 527 452}%
    \special{pa 427 427}%
    \special{pa 527 402}%
    \special{pa 527 452}%
    \special{fp}%
    \graphtemp=\baselineskip\multiply\graphtemp by -1\divide\graphtemp by 2
    \advance\graphtemp by .5ex\advance\graphtemp by 0.427in
    \rlap{\kern 0.750in\lower\graphtemp\hbox to 0pt{\hss $\theta_{3}$\hss}}%
    \special{pa 1427 73}%
    \special{pa 2073 73}%
    \special{fp}%
    \special{sh 1.000}%
    \special{pa 1973 48}%
    \special{pa 2073 73}%
    \special{pa 1973 98}%
    \special{pa 1973 48}%
    \special{fp}%
    \graphtemp=\baselineskip\multiply\graphtemp by -1\divide\graphtemp by 2
    \advance\graphtemp by .5ex\advance\graphtemp by 0.073in
    \rlap{\kern 1.750in\lower\graphtemp\hbox to 0pt{\hss $\theta_{4}$\hss}}%
    \special{pa 1427 427}%
    \special{pa 2073 427}%
    \special{fp}%
    \special{sh 1.000}%
    \special{pa 1527 452}%
    \special{pa 1427 427}%
    \special{pa 1527 402}%
    \special{pa 1527 452}%
    \special{fp}%
    \graphtemp=\baselineskip\multiply\graphtemp by -1\divide\graphtemp by 2
    \advance\graphtemp by .5ex\advance\graphtemp by 0.427in
    \rlap{\kern 1.750in\lower\graphtemp\hbox to 0pt{\hss $\theta_{5}$\hss}}%
    \hbox{\vrule depth1.000in width0pt height 0pt}%
    \kern 2.500in
  }%
}%

\begin{figure}
  \centerline{\box\graph}
  \caption{\label{fig:tetra}
  System diagram for the tetracycline model where $f_{1}$ is the
  concentration in the sampled compartment.  The other compartments do
  not have a physical interpretation.}
\end{figure}
(A mammary model was also fitted, as discussed in Section 5.4.)
It is assumed that the initial concentration in compartment 1 is 100\%
and that the only response measured is the concentration in compartment 1.

% Local Variables: 
% mode: latex
% TeX-master: "nraia2"
% End: 
