\appendix{Data Sets Used in Problems}

\section{BOD Data Set 2}

Data on biochemical oxygen demand (BOD) were obtained by
\citeasnoun{mars:1967} as described in Appendix 1, Section A1.3.
A second set of data is reported in Table~\ref{atbl:A4bod2}.
%\glossary{ Marske, D.}
\begin{table}[tbp]
  \caption{\label{atbl:A4bod2}
  Biochemical oxygen demand versus time.}
  \begin{center}
    \begin{tabular}{r r r r}
      \hline
      &\multicolumn{1}{c}{Biochemical}&&\multicolumn{1}{c}{Biochemical}\\
      &\multicolumn{1}{c}{Oxygen}&&\multicolumn{1}{c}{Oxygen}\\
      \multicolumn{1}{c}{Time}&\multicolumn{1}{c}{Demand}&
      \multicolumn{1}{c}{Time}&\multicolumn{1}{c}{Demand}\\
      \multicolumn{1}{c}{(days)}&\multicolumn{1}{c}{(mg/l)}&
      \multicolumn{1}{c}{(days)}&\multicolumn{1}{c}{(mg/l)}\\
      \hline
      1&0.47&5&1.60\\
      2&0.74&7&1.84\\
      3&1.17&9&2.19\\
      4&1.42&11&2.17\\
      \hline
    \end{tabular}
  \end{center}
\begin{quote}\small
  Copyright 1967 by D. Marske.  Reproduced from ``Biochemical Oxygen
  Demand Data Interpretation Using Sum of Squares Surface,'' M. Sc.
  Thesis, University of Wisconsin--Madison.  Reprinted with permission of
  the author.
\end{quote}
\end{table}

A model was derived based on exponential decay with a fixed rate
constant as
$$
f(x, \btheta ) =\theta_1
( 1 - e^{ \theta_2 x } )
$$
where $f$ is predicted biochemical oxygen demand and $x$ is time.

\section{Nitrendipene}

Data on binding of [$^{3}$H] nitrendipine to sites in rat
heart homogenate were obtained by \citeasnoun{abdo:1986}.
%\glossary{ Abdollah, S.}
In this study, experiments were performed to investigate the competition
for binding to the sites between nitrendipene (NTD), a calcium channel
antagonist, and nifedipine (NIF), another calcium channel antagonist.
Heart tissue was homogenated and incubated with radioactively tagged
NTD at molar concentration
$\approx5 \times 10^{-10}$ in the presence of different
concentrations of NIF, which are given in Table A4.2 as
\begin{table}
  \caption{\label{atbl:ntd}
  Radioactivity versus molar concentration of nifedipene for four
  tissue samples.}
  \begin{center}
    \begin{tabular}{r r r r r}
      \hline
      &\multicolumn{1}{c}{Counts$^{a}$}\\
      \multicolumn{1}{c}{$x=\log_{10}$(NIF)$^{b}$}&
      \multicolumn{1}{c}{Tissue Sample 1}&\multicolumn{1}{c}{2}&
      \multicolumn{1}{c}{3}&\multicolumn{1}{c}{4}\\
      \hline
      (0)&6696&4403&6133&5327\\
      (0)&6211&5042&5688&6274\\
      (0)&6385&*&6544&5210\\
      $-11$&6396&5259&6783&6811\\
      $-11$&6283&5598&6194&6751\\
      $-11$&6071&*&6188&7289\\
      $-10$&6545&4868&5674&7214\\
      $-10$&6378&4769&5583&5652\\
      $-10$&5932&*&6027&5700\\
      $-9$&5509&3931&5458&6184\\
      $-9$&6573&4503&5482&5175\\
      $-9$&5932&*&5878&5802\\
      $-8$&4763&2588&4173&3582\\
      $-8$&5389&3089&3837&7021\\
      $-8$&4131&*&4852&4187\\
      $-7$&4583&4359&3838\\
      $-7$&3815&3665&3936&3273\\
      $-7$&3539&*&4468&4562\\
      $-6$&3211&2149&3110&4004\\
      $-6$&4263&2216&3860&3520\\
      $-6$&3537&*&4297&4581\\
      $-5$&*&1433&3471&3719\\
      $-5$&*&1926&3674&2915\\
      $-5$&*&*&3990&4504\\
      (0)&*&*&5938&6396\\
      (0)&*&*&5948&6071\\
    \end{tabular}
  \end{center}
  $^{a}$ Total counts for $5 \times 10^{-10}$ Molar NTD additive.
  (Missing values coded as *.)\\
  $^{b}$ (0) means NIF concentration = 0.\\
  \begin{quote}\small
    Copyright 1986 by S. Abdollah.
    Reproduced from
    ``The Effect of Doxorbicin on the Specific Binding of
    [$^3 H$] Nitrendipenes to Rat Heart Microsomes,''
    M. Sc. Thesis, Queen's University.
    Reprinted with permission of the author.
  \end{quote}
\end{table}
$x= \log_{10} ( {\rm NIFconcentration})$, except for
the rows with (0), for which the actual concentration was 0.
The NIF has greater binding ability and so displaces the NTD.
Counts on radioactive material were obtained to determine how much
material was bound under different conditions.
When the NIF concentration is 0, all of the radioactive NTD is bound to
the sites, and so a large count is recorded:
as the NIF concentration increases,
it displaces NTD and so lower counts are recorded.
Although the nominal NTD concentration was $5 \times 10^{-10}$,
the actual concentrations were 4.76, 5.11, 4.78, and 5.02
$\times 10^{-10}$ respectively, for the four tissue samples.

The proposed model is
\begin{displaymath}
  f(x,\btheta)=\theta_1+\frac{\theta_2}{1+\exp[-\theta_4(x-\theta_3)]}
\end{displaymath}
where $f$ is the predicted total count and $x$ is $\log_{10}$(NIF
concentration).

\section{Saccharin Data Set 2}

Data on the concentration of saccharin in plasma were reported in
\citeasnoun{renw:1982} and are reproduced in Table A4.3.
%\glossary{ Renwick, A.G.}
\begin{table}
  \caption{\label{atbl:A4sac2}
    Saccharin concentration in plasma versus time.}
  \begin{center}
    \begin{tabular}{ r r r r}
      \hline
      &Saccharin&&Saccharin\\
      Time&Conc.&Time&Conc.\\
      (min)&($\mu$g/ml)&(min)&($\mu$g/ml)\\ \hline
      0&0&60&14.1\\
      5&184.3&75&8.0 \\
      15&102.0&90&5.7 \\
      30&50.5&105&4.0\\
      45&24.9&120&2.9\\ \hline
    \end{tabular}
  \end{center}
  \begin{quote}
    From ``Pharmacokinetics in Toxicology'' by A. G. Renwick, in {\em
      Principles and Methods of Toxicology}, A. Wallace Hayes, Ed.,
    Raven Press, 1982.  Reprinted with permission of the publisher.
  \end{quote}
\end{table}

% Local Variables: 
% mode: latex
% TeX-master: "nraia2"
% End: 
