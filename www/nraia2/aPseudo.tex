\appendix{Pseudocode for Computing Algorithms}

At several points in this book we have described how to compute
estimates and summary results.  To aid the reader in understanding the
computing methods and to provide guidance in implementing them, we
present pseudocode for the more important algorithms in this appendix.
\section{Nonlinear Least Squares}

The Gauss--Newton algorithm for nonlinear least squares with a
relative offset convergence criterion can be expressed in pseudocode
as:
\begin{verbatim}
initialize iteration counter(0) and step factor $\lambda=1$
repeat \{
    increment iteration counter
    error exit if maximum number of iterations exceeded
    evaluate residuals and derivatives
    decompose derivatives as $\bQ \bR$
    error exit if $\bR$ is computationally singular
    overwrite residuals by $\bQ \trans \bz$
    solve for $\bdelta$
    evaluate convergence criterion
    normal exit if (criterion $<$ tolerance)
    repeat \{
        evaluate residuals at $\theta+\lambda \bdelta$
        break loop if ($S ( \btheta+\lambda \bdelta )S( \btheta )$)
        halve $\lambda$
        error exit if ($\lambda$ $<$ minimum allowed)
    \}
    overwrite $S( \btheta )$ with $S( \btheta+\lambda \bdelta )$
    overwrite $\btheta$ with $\btheta+\lambda \bdelta$
    double $\lambda$ to maximum of 1
\}
\end{verbatim}
We implement this algorithm in three high level languages, {\bf
S},\cite{beck:cham:wilk:1988},
%\glossary{ Becker, R.A.}
%\glossary{ Chambers, J.M.}
%\glossary{ Wilks, A.R.}
{\bf GAUSS}\cite{edle:jone:1986},
%\glossary{ Edlefsen, L.E.}
%\glossary{ Jones, S.D.}
and {\bf SAS/IML}\cite{SAS:1985:iml} in the following subsections.

\subsection{Implementation in {\bf S}}

One of the fundamental issues to be resolved in any implementation of
nonlinear algorithms is the method of packaging the model function and
the independent variables.  In {\bf S} we can use default
values for arguments of functions for this.  For example, we define
the Michaelis--Menten model with the Puromycin data as:
\begin{verbatim}
Puromycin  function(theta = (205., 0.08), derivs = F,
    conc = (0.02, 0.02, 0.06, 0.06, 0.11, 0.11, 0.22,
    0.22, 0.56, 0.56, 1.1, 1.1), rate = (76., 47., 97.,
    107., 123., 139., 159., 152., 191., 201., 207., 200.))
\{
    denom  theta[2] + conc
    res  rate -- (theta[1] * conc/denom)
    if(!derivs) return(res)
    grad  matrix((conc/denom,  -- theta[1] * conc/denom\^{}2), ncol = 2)
    list(residual = res, gradient = grad)
\}
\end{verbatim}


The function {\bf Puromycin} has four arguments ({\bf theta},
{\bf derivs}, {\bf conc}, and {\bf rate}) but the only arguments which
are changed between calls are {\bf theta}, the model parameters, and
{\bf derivs}, a logical value which indicates if derivatives
are to be calculated.  Any actual arguments specified for these two
formal arguments will override the default values.  Using
{\bf conc}, the observed concentrations, and
{\bf rate}, the observed reaction rates, as arguments is a
convenient way of incorporating information about the observed
response and the regressor variables with the expressions which define
the residuals and the derivatives of the model function.

The Gauss--Newton nonlinear least squares algorithm is also specified
as a function with default arguments to specify characteristics such
as the maximum number of iterations and the minimum step factor.  This
function, which we call {\bf nlsfit}, is:
\begin{verbatim}
nlsfit  function(model, theta = eval(model[["theta"]]), maxiter = 25 * P,
    minfactor = 1/2\^{}10, tolerance = 0.001, verbose = T)
\{
    resid  model(theta, derivs = F)
    newssq  sum(resid\^{}2)
    P  length(theta)
    N  length(resid)
    ndof  N -- P
    mult  sqrt(ndof/P)
    iteration  0
    stepfactor  1.
    repeat \{
        iteration  iteration + 1
        if(iteration $>$ maxiter)
            stop("Maximum number of iterations exceeded")
        oldssq  newssq
        resgrad  model(theta, derivs = T)
        qrstr  qr(resgrad\$gradient)
        if(qrstr\$rank!=P) stop("Singular derivative matrix")
        qrslv  qr.coefqty(qrstr, resgrad\$residual)
        incr  qrslv\$coef
        converge  mult * sqrt(sum(qrslv\$qty[1:P]\^{}2)
            /sum(qrslv\$qty[ -- (1:P)]\^{}2))
        if(verbose) cat(iteration, "$<$", converge, "$>$", incr, fill = T)
        if(converge $<$ tolerance) break
        repeat \{
            trial  theta + stepfactor * incr
            if(stepfactor $<$ minfactor)
                stop("Step factor reduced below minimum")
            newssq  sum(model(trial, derivs = F)\^{}2)
            if(verbose) cat("  ", stepfactor,
                "(", newssq, ")", trial, fill = T)
            if(newssq $<$= oldssq) break
            stepfactor  stepfactor/2
        \}
        stepfactor  min(1., 2 * stepfactor)
        theta  trial
    \}
    list(model = model, coef = theta, qr = qrstr,
        residuals = resgrad\$residual, criterion = converge)
\}
\end{verbatim}


The name {\bf nlsfit} is chosen to correspond to
{\bf lsfit}, the {\bf S} function for fitting
linear regression models.  Furthermore, the object returned by
{\bf nlsfit} is similar to the object returned by {\bf lsfit}, so that
other functions in {\bf S} for
summarizing the results of a linear least squares fit can be used for
linear approximation summaries of the nonlinear least squares fit.

The {\bf S} code very closely follows the pseudocode given
above.  An initial evaluation of the residuals at the starting values
for {\bf theta} is used to set the sum of squares for later
comparisons.  The length of this residual vector is the number of
observations, {\bf n}, which, along with the number of
parameters, {\bf p}, is used to calculate a multiplier,
{\bf mult}, for the convergence criterion.  The
{\bf S} function {{\bf qr} performs the $QR$
decomposition of a matrix, and the function {\bf qr.coefqty}
returns both the coefficient vector, which is $\bdelta$ in our case,
and $\bQ \trans \by$, which is $\bQ \trans \bz$ in our case.  The
convergence criterion is calculated from the sum of squares of the
first $P$ elements of $\bQ \trans \bz$ (using the notation {\bf 1:P},
and the sum of squares of everything but the first $P$ elements [using the
notation {\bf -(1:P)}.

If the argument {\bf verbose} has the value T (true),
diagnostic output is produced by the function {bf cat}
during each iteration of the outer loop and each iteration of the
inner loop.  Diagnostic output for the Puromycin data appears as:
\begin{verbatim}
PUR.out <- nlsfit(Puromycin)
1 < 2.8046 > 8.02889 -0.0171078
     1.0000 ( 1205.66 ) 213.029 0.0628922
2 < 0.204279 > -0.425519 0.00109553
     1.0000 ( 1195.48 ) 212.603 0.0639878
3 < 0.0103471 > 0.0720588 0.000120553
     1.0000 ( 1195.45 ) 212.675 0.0641083
4 < 0.00100142 > 0.0075058 1.17256e-05
     1.0000 ( 1195.45 ) 212.683 0.06412
5 < 9.65585e-05 > 0.000725621 1.13078e-06
\end{verbatim}

The output from the outer loop consists of the iteration numbers at
the left margin, followed by the convergence criterion, in angle
brackets, followed by the requested increment.  The output from the
inner loop is indented and consists of the step factor, $\lambda$,
followed by the residual sum of squares, in parentheses, followed by
the parameter values.

\subsection{Implementation in GAUSS}

The principal input to the GAUSS {\bf lsfit}
procedure (proc) is the model function, which we code as in the
{\bf S} implementation, with two notable differences.
First, the residuals and the derivative matrix are returned as an $N
\times ( P+1 )$ matrix, because the current version of GAUSS
(1.49b, revision 19) allows a proc to return only one matrix.  Second,
the raw data are specified as global variables so that they are
available for other purposes.  With these conditions, we define the
Michaelis--Menten model with the Puromycin data as:
\begin{verbatim}
P\_MYCIN.PRG;
let conc = .02 .02 .06 .06 .11 .11 .22 .22 .56 .56 1.1 1.1 ;
let rate =  76  47  97 107 123 139 159 152 191 201 207 200 ;
proc p\_mycin( theta, derivs);
    local denom, resid, grad;
    denom = theta[2,1] + conc;
    resid  = rate  -- (  theta[1,1] *  conc./denom );
    if not derivs;
        retp(resid);
    endif;
    grad = conc./denom \~{} --theta[1,1] * conc./denom .\^{}2;
    retp(resid\~{}grad);
endp;

\end{verbatim}

The GAUSS procedure {\bf nlsfit} is similar to the {\bf S} routine, but
because GAUSS has no facilities for defining default arguments, we
include the number of iterations,
the minimum step factor, and the tolerance in the calling sequence.  The
procedure {\bf nlsfit} returns the $P+2$ vector
{\bf (theta$|$ssq$|$criterion)}, (the least squares estimates,
the sum of squares, and the convergence criterion, which is awkward to
compute outside the procedure).  If convergence is not achieved, an
error code can be returned by using the error function.  Diagnostic
output is written when {\bf verbose} is nonzero by calling
the procedure {\bf diagnost}.  If desired, the residuals and
derivative matrix could be returned to global variables by means of
the {\bf carput} function.  The procedure is:
\begin{verbatim}
proc nlsfit(\&model, theta, maxiter, minstep, tol, verbose);
local model:proc, mult, iter, stepsize, resgrad, res, incr, fullrank,
    criterion, ssq, trial, newssq;
res  = model( theta, 1);
nobs = rows( res );
npar = rows( theta );
mult = sqrt( (nobs -- npar) ./ npar );
iter = 0;
stepsize = 1;
do while 1;
    iter = iter + 1;
    resgrad = model( theta, 1);
    ssq = sumc( resgrad[ .,1] .\^{} 2 );
    gosub qr\_solv;
    pop incr;pop criterion;pop fullrank;
    if verbose;
        call diagnost(iter$|$"$<$"$|$criterion$|$"$>$"$|$incr,1);
    endif;
    if not fullrank;
        "Singular derivative matrix";stop;
    elseif criterion $<$ tol;
        retp(theta$|$ssq$|$criterion);
    elseif iter $>$ maxiter;
        "Maximum iterations reached";stop;
    endif;
    gosub step;
    pop stepsize;pop theta;
endo;
\end{verbatim}

The subroutine {\bf step}, which follows, computes the new
stepsize and
the corresponding parameter  vector and  returns them  to the  body of
the procedure, where they are ``pop''ed into the appropriate variables.
The code is:
\begin{verbatim}
step:
  do while 1;
    trial = theta + stepsize * incr;
    newssq = sumc( model( trial, 0 ) .\^{} 2 );
    if verbose;
        call diagnost(stepsize$|$"("$|$newssq$|$")"$|$trial,0);
    endif;
    if stepsize $<$ minstep;
        " Stepsize reduced below minimum ";stop;
    elseif newssq $<$ ssq;
        return( trial, minc( 1 $|$ 2 * stepsize ));
    endif;
    stepsize = stepsize ./ 2 ;
  endo;
\end{verbatim}



All that remains is to compute the $QR$ decomposition in the subroutine
{\bf r\_solv}.  GAUSS implements $QR$
decompositions via the LINPACK routines \cite{dong:bunc:mole:stew:1979}
%\glossary{ Dongarra, J.J.}
%\glossary{ Bunch, J.R.}
%\glossary{ Moler, C.B.}
%\glossary{ Stewart, G.W.}
DQRDC and DQRSL using the
oadexe/callexe\rm%
(pp commands.  For iterative applications
it is more efficient to preload the LINPACK code at startup:
for discussion of how to do this, see the December 1986 GAUSS
newsletter, vol. 2 no. 6.  Code for the subroutine
r\_solv\rm%
(pp using the allexe\rm%
(pp command is:
\begin{verbatim}
qr\_solv:
  local flag, qraux, work, jpvt, qty, dum, job, info, grad,
      beta, karg, ss\_tan, ss\_orth;
  flag = 1;
  qraux = zeros( npar , 1 );
  work = qraux;
  jpvt = qraux;                      @--- all columns free ---@
  qty = zeros( nobs, 1 );
  dum = 0;
  job = 100;                         @--- compute beta and Q'y---@
  info = 0;
  grad = submat(resgrad,0,seqa(2,1,npar))';
  res = resgrad[ .,1];
  callexe  gqrdc(grad,nobs,nobs,npar,qraux,jpvt,work,flag);
  beta = abs( diag( trim( grad', 0, nobs--npar)));
  karg = sumc( beta .$>$ beta[1,1] * 1e--14 );
  beta = zeros(npar,1);
  callexe
      gqrsl(grad,nobs,nobs,karg,qraux,res,dum,qty,beta,dum,dum,job,info);
  beta = submat( sortc( beta \~{} jpvt, 2 ), 0, 1 );
  ss\_tan = sumc( submat( qty, seqa(1,1,npar), 0 ) .\^{}2 );
  ss\_orth = sumc(submat( qty, seqa( npar+1, 1, nobs -- npar ), 0 ) .\^{}2 );
  return( karg == npar, mult*sqrt(ss\_tan./ss\_orth) , beta);
 endp;
end;
\end{verbatim}
With these procs, the program
lsfit(\&p\_mycin,205$|$.08,10,.001,.001,1)\rm%
(pp; produces the
output:
\begin{verbatim}
--------------Nonlinear estimation--------------
 
Starting values:     205.00000000       0.08000000
01 <    2.804604 >    8.028894   -0.017108
    1.000000 ( 1205.661845 )  213.028894    0.062892
02 <    0.204279 >   -0.425519    0.001096
    1.000000 ( 1195.477124 )  212.603375    0.063988
03 <    0.010347 >    0.072059    0.000121
    1.000000 ( 1195.449080 )  212.675434    0.064108
04 <    0.001001 >    0.007506    0.000012
    1.000000 ( 1195.448817 )  212.682940    0.064120
05 <    0.000097 >    0.000726    0.000001
\end{verbatim}

\subsection{Implementation in SAS/IML}

The principal input to the SAS/IML module
lsfit\rm%
(pp is the model function, which we code in a
submodule called odel\rm%
(pp.  This submodule includes the
data and code for the residuals and the derivatives.  For the
Michaelis--Menten model with the Puromycin data, we have:
\begin{verbatim}
start model(theta,res,grad,derivs);
 conc = \{ .02,.02,.06,.06,.11,.11,.22,.22,.56,.56,1.1,1.1\};
 rate = \{ 76,47,97,107,123,139,159,152,191,201,207,200\};
 denom = theta(|2,1|) + conc;
 res = rate --(theta(|1,1|) * conc / denom );
 if derivs = 1 then do;
 grad = (conc/denom)||(--theta(|1,1|)*conc/(denom \#\# 2));
 end;
finish;
\end{verbatim}

In SAS/IML, the $QR$ decomposition is achieved by means of the
{\bf GSORTH} function, which returns the $\bQ$ and $\bR$ matrices.
The increment is determined using the solve function,
{\bf ncr=solve(rhat,resid)}, and the convergence criterion
is calculated by invoking the Pythagorean decomposition of the total
sum of squares.

Like GAUSS, SAS/IML lacks an explicit repeat/break
structure, and so we use a {\bf until}
loop to calculate the
step factor.  The condition {\bf newssq $<$ oldssq}
is evaluated at
the end of the loop, so it is not necessary to initialize
{\bf newssq}.  The step factor, however, is half the appropriate
value on exit from the loop, so we multiply by 4, to a maximum of 1.

The main loop is implemented as fHdo iter = 1 to maxfP.  On
exit from the loop a message is printed and the program stops.  When
convergence is obtained, the program links directly to the finish
statement and returns.  One feature worth noting is that both the
residuals and the derivatives are initialized to their starting values
at the top of the module.  For each new iteration they are
recalculated using the new parameters at the end of the main loop,
saving a function evaluation.

If the verbose option is chosen, the fHreset nonamefP option
is set, and must be reset to its previous state after exit from the
module.  Diagnostic output is similar to that from the
{\bf S} function.  Code for the {\bf nlsfit} module is:
\begin{verbatim}
run model( theta, resid, grad, 1);
p = nrow( theta );
n = nrow( resid );
ndof = n -- p;
mult = sqrt( ndof / p );
Stepsize = 1;
do iter = 1 to maxiter;
    oldssq = ssq(resid);
    call gsorth( qhat, rhat, rank, grad);
    if rank = 1 then do;
        print "singular derivative matrix";
        stop;
    end;
    tan = qhat` * resid;
    ss\_tan = ssq( tan );
    incr = solve( rhat, tan );
    criterion = mult * sqrt( ss\_tan /( oldssq -- ss\_tan ) );
    if verbose then
        print iter($|$format=2.0$|$) criterion (incr`);
    if criterion $<$ tol then link returns;
    do until(newssq $<$ oldssq);
        if stepsize $<$ minstep then do;
            print "step factor reduced below minimum";
            stop;
        end;
        trial = theta + stepsize * incr;
        run model( trial, resid, grad, 0);
        newssq = ssq( resid );
        if verbose then
            print stepsize newssq (trial`);
        stepsize = stepsize / 2;
    end;
    theta = trial;
    stepsize = min( 1 $|$$|$ 4 * stepsize );
    run model( theta, resid, grad, 1);
end;
print " Maximum number of iterations reached: program terminated";
stop;
returns: finish;
\end{verbatim}

With these modules, the program
\begin{verbatim}
title NLSFIT to puromycin data;
PROC IML;
start model(theta,res,grad,derivs);
 \ldots
start nlsfit( theta, criterion, maxiter, minstep, tol, verbose);
 \ldots
theta = \{ 205, 0.08 \};
run nlsfit(theta,criterion,10,.001,.001,1);
reset name;print theta;
\end{verbatim}

produces the following output:
\begin{verbatim}
 NLSFIT to puromycin data
 
  1     2.8046    8.0289  -0.0171
1.0000    1205.7     213.0   0.0629
  2     0.2043   -0.4255 .0010955
1.0000    1195.5     212.6   0.0640
  3     0.0103    0.0721  1.2E-04
1.0000    1195.4     212.7   0.0641
  4   .0010014  .0075058  1.2E-05
1.0000    1195.4     212.7   0.0641
  5    9.7E-05   7.3E-04  1.1E-06
 
          THETA     COL1
 
          ROW1     212.7
          ROW2    0.0641
\end{verbatim}

\section{Linear Summaries and Studentized Residuals}

It is straightfoward to calculate the linear summary for the parameter
estimates using the method described in Section 1.2.3.  Pseudocode for
this is:
\begin{verbatim}
set $s=\sqrt{S( \hat \btheta )/(N-P)}$
calculate $\bR_1^{-1$}
calculate the length of each rows of $\bR_1^{-1$}
divide each row of $\bR_1^{-1$} by its length to produce $\bL$
multiply each row length by $s$ to give the parameter standard errors
form the correlation matrix, $\bL \bL \trans$
\end{verbatim}

To calculate the studentized residuals we need the diagonal of the
%.q "hat matrix"
$\bQ_1 \bQ_1 \trans$, but this is just the squared length of the
rows of $\bQ_{1}$.  Pseudocode for calculating the studentized
residuals is:
\begin{verbatim}
form the matrix $\bQ_{1}$ by applying the Householder transformations
  which define $\bQ$ to the first $P$ columns of the $N \timesN$ identity matrix
sum the squares of the elements of the rows of $\bQ_{1}$ to get $h_{nn}$
divide the $n$th residual by $s \sqrt{1-h_{nn}}$
\end{verbatim}

It is convenient to combine the two operations of calculating the
linear summary values and the studentized residuals into a single
function, say s.summary\rm%
(pp, which acts on the object
returned by lsfit\rm%
(pp.
\section{Multiresponse Estimation}

The algorithm for optimizing the multiresponse criterion is similar to
the nonlinear least squares algorithm given in Section A3.1, the main
difference being in calculation of the increment at each step.
Pseudocode is:
\begin{verbatim}
initialize iteration counter(0) and step factor $\lambda=1$
repeat \{
    increment iteration counter
    error exit if maximum number of iterations exceeded
    evaluate residual matrix and derivative array
    calculate $| \bZ \trans \bZ |$, $\bomega$, and $\bOMEGA$
    attempt the Cholesky decomposition $\bOMEGA=\bC \trans \bC$
    if ($\bOMEGA$ is not positive definite) then \{
        criterion is LARGE (note 1)
        decompose as $\bOMEGA=\bU \bD \bU \trans$ (note 2)
        $\bdelta=- \bU ( \bD+2d_1 \bI )^{-1} \bU \trans \bomega$
    \} else \{
        solve $\bC \trans \bC \bdelta=- \bomega$ for $\bC \bdelta$
        evaluate convergence criterion
        solve for $\bdelta$
    \}
    normal exit if (criterion $<$ tolerance)
    repeat \{
        evaluate $\bZ$ at $\btheta+\lambda \bdelta$
        break loop if ($| \bZ \trans \bZ |$previous value)
        halve $\lambda$
        error exit if ($\lambda$ $<$ minimum allowed)
    \}
    overwrite previous $| \bZ \trans \bZ |$ with current value
    overwrite $\btheta$ with $\btheta+\lambda \bdelta$
    double $\lambda$ to maximum of 1
\}
\end{verbatim}

Notes:
\begin{enumerate}
\item When $\bOMEGA$ is not positive definite, the convergence
  criterion is set to a predetermined value which is greater than the
  tolerance, so convergence cannot be declared: this is because the
  determinant cannot be minimized at this point unless $\bOMEGA$ is
  positive definite.
\item The decomposition $\bOMEGA=\bU \bD \bU \trans$ is an
  eigenvalue--eigenvector decomposition of the symmetric matrix
  $\bOMEGA$.  The diagonal matrix $\bD$ has diagonal entries $d_1 \le
  d_2 \le\ldots\le d _{P}$, while $\bU$ is a $P\times P$ orthogonal matrix.
\end{enumerate}

The calculation of $| \bZ \trans \bZ |$, $\bomega$, and $\bOMEGA$ from the
residual matrix and the gradient array is as described in Section
4.2.2.  The pseudocode is
\begin{verbatim}
decompose $\bZ$ as $\bQ \bR$
evaluate $| \bZ \trans \bZ |$ as the square of the product of the diagonal elements of $\bR_{1}$
for $p=1,2 ,\ldots, P$ \{
    overwrite $\bZ_{{(p)}}$ by $\bG_{p=\bQ} \trans \bZ_{{(p)}}\bR_1^{-1$}
    calculate $\lb \bomega \rb_{p}$ from diagonal of $\bG_{p}$
    for $q=1,2 ,\ldots,p$ \{
        calculate $\lb \bOMEGA \rb_{pq}$ as in (4.24)
        store $\lb \bOMEGA \rb_{qp=\lb} \bOMEGA \rb_{pq}$
    \}
\}
\end{verbatim}
FORTRAN implementations of these algorithms are given in \{.bates watts
algorithm 1984.\}.
%\glossary{ Bates, D.M.}
%\glossary{ Watts, D.G.}
\par\vspace{0.0\baselineskip}
\section{Linear Systems of Differential Equations}

Pseudocode for the methods described in Appendix 5 when $\bA$ is
diagonalizable is given here.  Extensions for nondiagonalizable $\bA$
or complex eigenvalues are obtained from the formulas in Sections A5.2
and A5.3.

Before presenting the pseudocode, however, we present a compact way of
specifying the form of the system of linear differential equations and
the way that the components of $\btheta$ enter the system.  We define a
parameter use matrix $\bJ$ as described in Bates and Watts (1985,
Section 3.3).
%{.bates watts 1985 techno special.}
%\glossary{ Bates, D.M.}
%\glossary{ Watts, D.G.}
The matrix $\bJ$ consists of three columns: the entry in the first
column is the parameter number, the entry in the second column is the
source compartment, and the entry in the third column is the sink
compartment.  If the parameter is an initial condition instead of a
rate constant, the value --1 is used in the third column: if the
parameter is a dead time, the value 0 is used in the second and third
column.  For Example $\alpha$-Pinene 3, which involves only rate
constants, the parameter use matrix is
$$
\bJ =
\left[ \matrix{
  \matrix{1 \cr 2 \cr 3 \cr 4 \cr 5}
  \matrix{1 \cr 1 \cr 3 \cr 3 \cr 5}
  \matrix{2 \cr 3 \cr 4 \cr 5 \cr 3}
}\right]
$$
For Example $\alpha$-Pinene 9, where the reaction path from 3 to 4 is
dropped,
$$
\bJ =
\left[ \matrix{
  \matrix{1 \cr 2 \cr 3 \cr 4}
  \matrix{1 \cr 1 \cr 3 \cr 5}
  \matrix{2 \cr 3 \cr 5 \cr 3}
}\right]
$$
The last two columns of the parameter use matrix correspond to the
first two columns of the parameter summary tables which we used in
these examples.

In Example Tetracycline 2, the third parameter is the initial
condition in compartment 1, so the parameter use matrix is
$$
\bJ =
\left[ \matrix{
   \matrix{1 \cr 2 \cr 3}
   \matrix{1 \cr 2 \cr 1}
   \matrix{2 \cr 0 \cr -1}
}\right]
$$
When dead time is involved, as in Example Tetracycline 4, the source
and sink compartments are coded as 0, so the parameter use matrix is
extended to
$$
\bJ =
\left[ \matrix{
   \matrix{1 \cr 2 \cr 3 \cr 4}
   \matrix{1 \cr 2 \cr 1 \cr 0}
   \matrix{2 \cr 0 \cr -1 \cr 0}
}\right]
$$

Pseudocode to generate the values of $\bA$,
$\bgamma_{0}$,
$t_{0}$,
and
$\bA_{(p)}$,
$\bgamma_{(p)} (0)$, and
$\partial t_0 / \partial \theta_{p}$, $p=1 ,\ldots, P$,
from the parameter use matrix
$\bJ$, the parameter vector $\btheta$, and the fixed part of the
initial conditions, $\bgamma_{{{\rm }} fix}$, is
\begin{verbatim}
initialize $\bA$ to $ {\bf 0} $, $\bgamma_{0}$ to $\bgamma_{{{\rm }} fix}$
for $p=1 ,\ldots, P$
    initialize $\bA_{(p)$} to $ {\bf 0} $, $\bgamma_{(p)} (0)$ to $ {\bf 0} $
initialize $d t_0 / d \btheta \trans$ to $ {\bf 0} $
for $j=1 ,\ldots, J$ \{
    $p=\lb \bJ \rb_{j1}$
    $\phi=\exp ( \lb \btheta \rb_p )$
    $i=\lb \bJ \rb_{j2}$
    $k=\lb \bJ \rb_{j3}$
    if ($i=0$) \{
        $t_{0=\lb} \btheta \rb_{p}$
        $\lb d t_0 / d \btheta \trans \rb_{p=1$}
    \} else if ($k=-1$) \{
        increment $\lb \bgamma_0 \rb_{i}$ by $\phi$
        increment $\lb \bgamma_{(p)} (0) \rb_{i}$ by $\phi$
    \} else \{
        decrement $\lb \bA \rb_{ii}$ by $\phi$
        decrement $\lb \bA_{(p)} \rb_{ii}$ by $\phi$
        if ($k0$) \{
            increment $\lb \bA \rb_{ki}$ by $\phi$
            increment $\lb \bA_{(p)} \rb_{ki}$ by $\phi$
        \}
    \}
\}
\end{verbatim}
In this code, logarithms of the rate parameters and the initial
conditions are used, but the delay time parameter is not transformed.

Pseudocode to use these values to create the predicted responses $\bH$
and the derivatives $\bH_{(p),}p=1 ,\ldots, P$ at the times
$t_n ,n=1 ,\ldots, N$, is
\begin{verbatim}
decompose $\bA=\bU \bLAMBDA \bU^{-1$}
if (any complex eigenvalues) error exit
decompose $\bU$ with an $LU$ decomposition and check condition (note 1)
if (condition unacceptable) error exit
solve $\bU \bxi_{0=\bgamma}_{0}$ for $\bxi_{0}$
for $p=1 ,\ldots, P$
    solve $\bU \bC_{(p)=\bA}_{(p)} \bU$ for $\bC_{(p)$}
for $n=1 ,\ldots, N$ \{
    $\tau=t_{n-t}_{0}$
    if ($\tau0$) $\tau=0$
    for $k=1 ,\ldots, K$ \{
        $\lb \bxi \rb_{k=e}^{{\lambda}_k \tau}\lb \bxi_0 \rb_{k}$
        for $j=1 ,\ldots, K$ \{
            ${\rm dif=\lambda}_{j-\lambda}_{k}$
            if ($\tau{\rm dif\epsilon$)}
                $\lb \bB \rb_{kj=\taue}^{{\lambda}_k \tau}$
            else
                $\lb \bB \rb_{kj={e}^{{\lambda}_j \tau}-e^{\frac{\lambda}_k\tau}{\rm dif}$}
        \}
    \}
    store $\bgamma_{n=\bU} \bxi$
    for $p=1 ,\ldots, P$ \{
        $\bxi_{(p)} ( \tau )=e^{{LAMBDA} \tau}\bxi_{(p)} (0)+\tau_{(p)} e^{{LAMBDA} \tau} LAMBDA \bxi_{0}$
        increment $\bxi_{(p)} ( \tau )$ by the componentwise product
            of $\bC_{(p)$} and $\bB$ times $\bxi_{0}$
        store $\bgamma_{(p)n=\bU} \bxi_{(p)} ( \tau )$
    \}
\}
\end{verbatim}

Notes:
\begin{enumerate}
\item The $LU$ decomposition \cite[Chapter
  1]{dong:bunc:mole:stew:1979} of  $\bU$ with an 
%\glossary{ Dongarra, J.J.}
%\glossary{ Bunch, J.R.}
%\glossary{ Moler, C.B.}
%\glossary{ Stewart, G.W.}
estimate of the condition of the matrix is used to determine if $\bU$
is computationally singular and to solve linear systems based on $\bU$.
\end{enumerate}

\section{Profile Calculations}

\subsection{Generation of $\btau$ and the Profile Traces}

To generate the values of $\tau$ and the profile traces, we assume that
the components of $\hat \btheta$ and their approximate standard errors
are known.
For each component $\theta_{p}$, we generate $\tau ( \theta_p )$ and
the profile trace $\tilde \btheta_{-p} ( \theta_p )$ first to the left
of $\hat \theta_{p}$, then to the right.
We calculate the $\tau$ values until the absolute value exceeds
$\tau_{max=\sqrt{\FPNP}}$ where $\alpha$ is small, say $\alpha=0.01$.
In some cases, such as Example BOD 8 or Example Isomerization 5, the
value of $\tau$ approaches an asymptote which is less than
$\tau_{max}$, so we impose a condition on the maximum number of values
$k_{max}$ to be calculated on either side of $\hat \theta_{p}$, say
$k_{max} = 30$  A nominal step size of se$( \hat \theta_p ) /$step 
with step$=8$ is used to start the process, but thereafter the
step size is determined from the slope of the curve $\tau$ versus $t$,
with limits to prevent the step size from becoming too large.

Pseudocode for the calculation is:
\begin{verbatim}
for $p=1 ,\ldots, P$ \{
    $\Delta=- {\rm se} ( \hat \theta_p ) /$'-1p'step
    $t=0$
    repeat \{
        invslope$=1$
        for $k=1 ,\ldots, k_{max}$ \{
            $t=t+$invslope
            minimize $S ( \btheta )$ with $\theta_\hat {p=\theta}_{p+\Delta}
\timest$ obtaining $\tilde S ( \theta_p )$ and $\tilde \btheta_{-p$}
            invslope$=$abs$ \left(\frac{\tau\timess^2}{{\rm se\times\bz}\trans\bv_p}\right)$
            record $\tau ( \theta_p )={\rm sign(} \Delta )\times\sqrt{\tilde \hat S} / s$'-4p', $\btheta$, and invslope
            invslope$=$min(4,max(invslope,1/16))
            if (abs$( \tau )\tau_{max}$) break loop
        \}
        $\Delta=- \Delta$
        if ($\Delta0$) break loop
    \}
\}
\end{verbatim}

Minimizing $S ( \btheta )$ with
$\hat\theta_{p}=\theta_{p+\Delta}\times k$ is done with a few simple
modifications to the Gauss--Newton 
code.  In addition to evaluating the residuals and derivatives we
remove the $p$th column of the derivatives, then solve for the
increment $\bdelta_{-p}$ and form $\bdelta$ from $\bdelta_{-p}$ with
zero in the $p$th position.  Note that $\tilde \btheta_{-p}$ is used
as the starting value for the next iteration.
\subsection{Profile Pair Plots}

In producing the profile pair plots, we first generate a vector of
$\tau$ values, $\btau_{p},p=1 ,\ldots, P$, of length $n_{p}$ for
each parameter, and the corresponding $n_{p\times P}$ matrix $\bM_{p}$
of parameter values.  Each of these matrices is transformed to the
$\tau$ scale in the following steps:
\begin{verbatim}
for $p=1,2 ,\ldots, P$ \{
    store $s_{{\theta} \to \tau ,p}$, the interpolating spline for the $p$th
        column of $\bM_{p}$ as a function of $\btau_{p}$.
    store $s_{{\tau} \to \theta ,p}$, the interpolating spline for $\btau_{p}$
        as a function of the $p$th column of $\bM_{p}$
    for $q=1,2 ,\ldots, P$  and $q!=p$ \{
        $\bg_{pq}=s_{{\theta} \to \tau ,p} ( p$th column of $\bM_q )$
        overwrite $\bg_{pq}$ by arccos($\bg_{pq} / \btau_p )$
        store $s_{{t} \to g ,pq}$ the interpolating spline for
            $\bg_{pq}$ as a function of $\btau_{p}$
    \}
\}
\end{verbatim}

When determining the splines $s_{{\theta} \to \tau ,p}$ and $s_{{\tau}
\to \theta ,p}$ we include a zero entry in $\btau_{p}$ with a
corresponding row $\hat \btheta$ in $\bM_{p}$.  This entry and the
corresponding row must be eliminated before the division by $\btau_{p}$.

To interpolate the projection of the contours
$S ( \btheta )=S^i ,i=1 ,\ldots,m$, into the
$( \theta_p , \theta_q )$ plane, we convert the levels to the
$\tau$ scale as
$$
k^i=\sqrt{S^{i-S(} \hat \btheta )}/s
$$
and determine the angles for the points on the traces as
described in Appendix 6.
These four angle pairs are
\begin{eqnarray*}
  \bp_1&=&(0,s_{{t} \to g ,pq}(+k^i ))\\
  \bp_2&=&( \pi ,s_{{t} \to g ,pq}(-k^i ))\\
  \bp_3&=&(s_{{t} \to g ,qp}(+k^i ),0)\\
  \bp_4&=&(s_{{t} \to g ,qp}(-k^i ), \pi )
\end{eqnarray*}
We convert these angles to an average angle and a phase difference by
\begin{verbatim}
for $j=1 ,..., 4$\{
    $a_{j=(} \lb \bp_j \rb_1+\lb \bp_j \rb_2 )/2$
    $d_{j=\lb} \bp_j \rb_1-\lb \bp_j \rb_{2}$
    if $(d_{j0)$} \{
        replace $d_{j}$ by $- d_{j}$
        replace $a_{j}$ by $- a_{j}$
    \}
\}
\end{verbatim}
and for $d_{j}$ as a function of $a_{j}$, determine
$sp_{{a} \to d ,pqi}$, an interpolating spline with period $2\pi$.  A
sequence of $K$ points (usually $K$ is between 50 and 100) on the
interpolating contour is evaluated using
\begin{verbatim}
for $k=1 ,..., K$ \{
    $a=(k-1)\times2\pi /(K-1)-\pi$
    $d=sp_{{a} \to d ,pqi}(a)$
    $\tau_p=\cos (a+d/2)$
    $\tau_q=\cos (a-d/2)$
    $\theta_{p=s}_{{\tau} \to \theta ,p}( \tau_p )$
    $\theta_{q=s}_{{\tau} \to \theta ,q}( \tau_q )$
\}
\end{verbatim}
and plotted.

% Local Variables: 
% mode: latex
% TeX-master: "nraia2"
% End: 
