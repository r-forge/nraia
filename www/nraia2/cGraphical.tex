\chapter{Exact Inference Regions for Nonlinear Models}

\prologue{ ``What can we know?  or what can we discern, When error
  chokes the windows of the mind?''}{Sir John Davies}

In Chapter 2 we derived linear approximation inference regions for
parameters in nonlinear models.
Although these approximate regions are easy to compute and display, 
it is unfortunately true that  in many nonlinear analyses they
are woefully inadequate, and they give no information as to when they are adequate or not.
\index{linear approximation!inference region}
In this chapter we develop procedures for calculating exact 
inference regions using the likelihood, Bayesian and sampling theory
approach. 

\section{Likelihood Regions}

As we have seen in Section 2.2 and 2.4, the spherical normal assumption for the disturbance $\bZ$
in the model (\ref{eqn:nlmatmodel}) dictates that statistical inference using the
likelihood approach is directly related to the
geometry of the expectation surface in the response space.
For linear and nonlinear models with the spherical normal assumption,
a likelihood contour consists of all values
\index{likelihood!contour}
\index{contour!likelihood}
of $\btheta$ for which $\boeta ( \btheta )$ is a fixed
distance from $ \by$, that is, all $\btheta$ for which
$S ( \btheta )$ equals a constant.
And as in Section 1.3 and 2.3, we associate a confidence
level with a contour ({\em c.f.} equation \ref{eqn:nlikeS-Shat})  so that 
the nominal $1-\alpha$ joint likelihood region consists
\index{likelihood!region}
of all values of $\btheta$ for which
\begin{equation}
  S ( \btheta ) = S( \hat \btheta ) + Ps^2 \FPNP
  \label{eqn:ssc}
\end{equation}
Geometrically, as with the linear model, this is the intersection of the expectation
surface with a sphere centered at $\by$.
Now, however, the surface is not planar and there is no easy way
to map the points on the expectation surface back to the
parameter space even if we could determine those points on the
intersection.

When $P= 2$, we can determine a likelihood contour in $\btheta$
by standard contouring methods, that is, by evaluating
$S ( \btheta )$ for a grid of $\btheta$ values and
approximating the contour by straight line segments in the grid.

EXAMPLE:puro13 Puromycin likelihood contours
%%%\input{puro13}

EXAMPLE:bod7 BOD likelihood contours
%%%\input{bod7}

Determining the shape of badly distorted contours like those in
Figure~\ref{fig:BODlikcont} can take several tries.
We evaluated an initial grid of likelihood
values and found that the contour went beyond the grid,
so we changed the limits and reevaluated the grid, repeating the
procedure until we obtained a satisfactory plot.
The contours in Figure~\ref{fig:BODlikcont} illustrate another deficiency of
standard contouring methods.
The nominal 95\% contour appears to have two disjoint pieces---the main
body and a thin region centered near $\btheta = ( 40 ,  0 ) \trans$.
In fact, this contour should be one continuous curve, and the two pieces
are an artifact of the way that the curve is traced by a computer
program.
Contouring programs typically have difficulty with long, thin segments
of contours.

This discussion has focused on 2-parameter models, but
nonlinear models with many more parameters occur and, unfortunately,
standard contouring methods are not easily extended beyond $P=2$.
One approach for multiparameter models is to try to
evaluate the likelihood on a $P$-dimensional grid.
This can be expensive, since the amount of
computing effort and the amount of storage required for the grid
grows exponentially with $P$.
Also, the analyst must choose the bounds of the grid before evaluating
a contour, and
these bounds may not encompass the entire contour, or they may be so
wide that the resolution over the region of interest is poor.
In this case, the analyst would have to guess at a new set of bounds and
reevaluate the contour, thereby adding to the expense of the process.
Even when the grid is evaluated, approximation of the contour and
display of the approximation in many dimensions is difficult.

One way of avoiding multidimensional grids
is to evaluate the sum
of squares function on a series of 2-dimensional grids
corresponding to each pair of parameters.
This requires one grid for the $( \theta_1 , \theta_2 )$
pair, one grid for the $( \theta_1 , \theta_3 )$ pair, and
so on, for a total of $P ( P - 1 ) / 2$ grids.
Contours on the 2-dimensional grids can be easily calculated and
displayed, and from these contours the analyst can gain insight
into the multidimensional shape of the likelihood region.
However, it is not clear which likelihood or sum of squares
to evaluate at each point in these 2-dimensional grids.
Two choices are to evaluate the {\em conditional likelihood}
\index{conditional likelihood}
\index{likelihood!conditional}
function by varying a pair of parameters while
holding the others fixed at their least squares
estimates, or to evaluate the
2-dimensional {\em profile likelihood}
function by finding the minimum sum of squares over all the other
coordinates for each point on the grid.

Both these approaches have disadvantages.
On the one hand,
the conditional likelihood function does not always present
a comprehensive view of the likelihood contour, since it only
shows selected cross-sections of the contour, and
the global behavior of the contour can be quite different from
the sectional behavior.
On the other hand,
evaluating the profile likelihood requires solving a $( P - 2 )$-dimensional
nonlinear least squares problem for each of the
points on the $P ( P - 1 ) / 2$ grids, which could be computationally
expensive.
To mitigate these difficulties, we propose making profile $t$
plots and profile pair sketches as described in the next section.

\subsection{Profile $t$ Plots}

To develop marginal likelihood intervals for nonlinear model
parameters, we begin by relating a linear model interval to the sum
of squares function.
For a linear model,
a $(1 - \alpha)$ marginal interval for $ \beta_{p}$ can be written
in terms of the studentized parameter
\begin{displaymath}
\frac{\beta_{p}-\hat\beta_p}{{\rm se} ( \hat \beta_p )}
 =\delta ( \beta_p )
\end{displaymath}
as
\begin{displaymath}
- \tNP\le\delta ( \beta_p ) \le\tNP
\end{displaymath}
But the studentized parameter can also be written

ATAMO!!!!!!

\begin{equation}
  \frac{\beta_p-\hat\beta_p}{{{\rm se}} ( \hat \beta_p )} = {{\rm
  sign}} ( \beta_p - \hat \beta_p ) \sqrt { { \tilde S ( \beta_p ) - S (
  \hat{\bbeta}) }} / s
  \label{eqn:tratio}
\end{equation}
where
\begin{equation}
  \tilde S ( \beta_p ) = min_{{\beta}_{-p}} S ( ( \beta_p ,
  {\bbeta}_{-p} \trans ) \trans ) = S ( ( \beta_p , \tilde{\bbeta}_{-p}
  \trans ) \trans )
  \label{eqn:Stilde}
\end{equation}
is the profile sum of squares function and
$\tilde {\bbeta}_{-p} =
( \tilde \beta_1 ,\ldots, \tilde \beta_{p-1,} \tilde \beta_{p+1}
 ,\ldots, \tilde \beta_P ) \trans$
is the least squares estimate of ${\bbeta}_{-p}$
conditional on $\beta_{p}$, {\em i.e.} the tangent point on the contour at $\beta_p$.  
The notation $( \beta_p , \tilde {\bbeta}_{-p} \trans ) \trans$ indicates the
vector with elements
$( \tilde \beta_1 ,\ldots, \tilde \beta_{p-1,} \beta_p , \tilde \beta_{p+1}
 ,\ldots, \tilde \beta_P )$.

For a nonlinear model, we define the {\em profile t\/} function,
$\tau ( \theta_p )$, as
\index{profile $t$}
\begin{equation}
  \tau ( \theta_p )={{\rm sign}} ( \theta_p - \hat \theta_p )
  \sqrt{\tilde S ( \theta_p ) - S ( \hat \btheta )}/s
  \label{eqn:proft}  
\end{equation}
using the same notation.
By analogy with the linear model, we define a nominal
$(1 - \alpha)$ likelihood interval for
\index{likelihood!interval}
$\theta_{p}$ as the set of all $\theta_{p}$ for which
\begin{displaymath}
- \tNP\le\tau ( \theta_p )\le\tNP
\end{displaymath}
The profile $t$ function is similar to the $\chi$ statistic used
by \citeasnoun{blis:jame:1966}.
%\glossary{ Bliss, C.I.}
%\glossary{ James, A.T.}

Plots of the profile $t$ function provide exact likelihood
intervals for individual parameters and, in addition, reveal how
nonlinear the estimation situation is.
To see this, suppose the model were linear.
Then a plot of $\tau ( \theta_p )$ versus
$\theta_{p}$ would be a straight line.
In particular, as seen from (\ref{eqn:tratio}), a plot of
$\tau ( \theta_p )$ versus the studentized parameter,
$\delta ( \theta_p )=( \theta_p-\hat\theta_p ) /
{{{\rm se}} ( \hat \theta_p )}$,
would be a straight line through the origin with unit slope.
For a nonlinear model, a plot of
$\tau ( \theta_p )$ versus $\delta ( \theta_p )$ will be curved,
which reveals valuable information about the behavior of the estimate of that parameter.
\index{nonlinearity!and profile $t$ plot}

EXAMPLE:puro14 Puromycin profile t plots
%%%\input{puro14}

EXAMPLE:bod8 BOD profile t plots
%%%\input{bod8}

\subsection{Profile Traces}

Another useful plot is the likelihood {\em profile trace\/}
obtained by
plotting the components of the conditional least squares vector
$\tilde \btheta_{-p}$ as a function of $\theta_{p}$.
\index{likelihood!profile trace}
\index{profile!trace}
For example, after evaluating the profile likelihood for $\theta_{1}$,
we can plot $ \tilde \theta_{2}$ versus $\theta_{1}$,
$\tilde \theta_{3}$ versus $\theta_{1}$, and so on,
up to $\tilde \theta_{P}$
versus $\theta_{1}$.
Next, we evaluate the profile likelihood for $\theta_{2}$
and plot $ \tilde \theta_{1}$ versus $\theta_{2}$,
$ \tilde \theta_{3}$ versus $\theta_{2}$, and so on,
up to $\tilde \theta_{P}$
versus $\theta_{2}$.
We continue to work through the parameters $\theta_3 $ to
$\theta_{P}$, calculating the conditional minima of the other
parameters, and plotting the profile traces.
Finally, we combine the plots of
$ \tilde \theta_{q}$ versus $\theta_{p}$ and
$ \tilde \theta_{p}$ versus $\theta_{q}$,
to generate the pairwise profile traces.
As before, it is convenient to do the calculations
using studentized parameters.

Plots of the profile traces provide useful information on how the
parameters interact.
For a linear model with studentized parameters, the profile traces
on a plot of $\delta ( \theta_q )$ versus $\delta ( \theta_p )$
consist of straight lines intersecting at
$( 0 ,  0 )$ with slopes of $\lb \bC \rb_{pq}$ for
the trace of $\tilde \delta ( \theta_q )$ on $\delta ( \theta_p )$
and $1/ \lb \bC \rb_{pq}$ for the trace of $\tilde \delta ( \theta_p )$
on $\delta ( \theta_q )$, where $\bC$
is the parameter estimate codependence matrix.
If the codependence between the parameter estimates is small, the
angle between the profile traces is $~90^\circ$.

For nonlinear models, the profile traces will be curved,
\index{nonlinearity!and profile traces}
the curving of the lines providing information on
how the parameter estimates affect one another and on
the shape of the projection of the likelihood
contours onto the $( \theta_p , \theta_q )$ plane.
If the contours are long and thin, the profile traces will
be close together; if the contours are fat, the profile
traces will tend to be perpendicular; and if the contours are
nearly elliptical, the profile traces will be straight.

EXAMPLE:puro15 Puromycin profile t plots
%%%\input{puro15}

EXAMPLE:bod9 BOD profile t plots
%%%\input{bod9}

\subsection{Profile Pair Sketches}
\index{profile pair sketches}

As stated earlier, it is not generally feasible to determine and plot
likelihood contours for models with several parameters.
However, we can use the profile sums of squares and the profile traces
to create very accurate approximations to the 2-dimensional
projections of the likelihood region and thus get a visual indication
of the extent of the region and the nonlinear dependence of parameter
estimates upon each other.
To determine the projections of the 95\% contour on the
$( \theta_1 , \theta_2 )$ plane, for example, we use
the profile sum of squares for $\theta_{1}$ to find where the contour
intersects the trace of $\tilde \theta_{2}$ on $\theta_{1}$.
This gives two points on the contour.
In addition, we know that the tangent to the contour must be vertical
at these points, since they represent the bounds of the contour in the
$\theta_{1}$ direction.
Similarly, from the profile sum of squares for $\theta_{2}$ and from
the trace of $\tilde \theta_{2}$ on $\theta_{1}$, we determine two
more points on the contour and we know that the contour will have
horizontal tangents at these points.

By using all of this information---the profile $t$ plots, the points on
the contour, the directions of the tangent to the contour at these
points, and the fact that the contour is bounded by the parameter
values at these points---we can create a very accurate interpolation
of a contour using the methods described in Appendix 6.  We call these
interpolated curves {\em profile pair sketches}.
\index{profile pair sketches}

EXAMPLE:puro16 Puromycin sketches 
%%%\input{puro16}

EXAMPLE:bod10 BOD sketches 
%%%\input{bod10}

When there are more than two parameters, a similar procedure
is followed to generate the sketches.
For example, with three parameters, for each of a set of values of
$\theta_{1}$ we converge to
$\tilde \btheta_{-1} = ( \tilde \theta_2 ( \theta_1 ),
\tilde \theta_3 ( \theta_1 )) \trans$,
to produce $\tilde S ( \theta_1 )$.
This information is used to calculate $\tau ( \theta_1 )$,
the coordinates of the profile traces
$\tilde \theta_2 ( \theta_1 ) $ and $\tilde \theta_3 ( \theta_1 )$,
and the coordinates of some of the points on the joint likelihood
contour for $ ( \theta_1 ,  \theta_2 )$
and for $ ( \theta_1 ,  \theta_3 )$.
Next, we choose a set of values for $\theta_{2}$ and converge
to $\tilde \btheta_{-2}=
( \tilde \theta_1 ( \theta_2 ),
\tilde \theta_3 ( \theta_2 )) \trans $,
this time producing $\tau ( \theta_2 )$,
the coordinates of the profile traces
$ \tilde \theta_1 ( \theta_2 )$
and
$ \tilde \theta_3 ( \theta_2 )$,
and the coordinates of some of the points on the joint
likelihood contour for $ ( \theta_1 ,  \theta_2 )$
and for $ ( \theta_2 ,  \theta_3 )$.
Then a range of values of $\theta_{3}$ is chosen and
we converge to
$\tilde \btheta_{-3} =
( \tilde \theta_1 ( \theta_3 ),
 \tilde \theta_2 ( \theta_3 ))$,
this time producing $\tau ( \theta_3 )$,
the coordinates of the profile traces
$ \tilde \theta_1 ( \theta_3 )$
and
$ \tilde \theta_2 ( \theta_3 )$,
and the coordinates of some of the points on the joint likelihood
contour for $ ( \theta_1 ,  \theta_3 )$
and for $ ( \theta_2 ,  \theta_3 )$.
We then plot the interpolated contours for each pair of parameters.
If the profile traces are fairly straight and distinct in both the
original and the $\tau$ coordinates, the contour will be fairly
elliptical and the sketches will be very accurate.
If the traces are curved in the original coordinates and tend to be
coincident in the $\tau$ coordinates over an appreciable range, then
the contours will be decidedly nonelliptical in the original
coordinates and the sketches may not be accurate.

EXAMPLE:iso2 Isomerization profile t plots for original parameterization
%%%\input{iso2}

The computations to generate profile $t$ plots and profile traces 
are very efficient because we start from the least squares estimates of the
previous calculation, and because the problem is always of
reduced dimension $P - 1$.
Also, at each value of the parameter of interest, we simultaneously
generate the profile $t$ value and the converged values of
the vector $\tilde \btheta_{-p}$, which also provides most of the data
necessary for sketching the profile pairs.
Profile $t$ functions can also be used to determine exact likelihood
intervals and bands for the expectation function.
For all these calculations, only minor modifications to standard
nonlinear regression software are necessary.

\subsection{ Likelihood Bands}

The profile $t$ function can be used to determine likelihood intervals
for the expectation function at any point $\bx_{0}$ by
reparametrizing the model function so that a new parameter, say
$\phi_{1}$, is the response at $\bx_{0}$.
The remaining parameters can be chosen as $\phi_p=\theta_{p}$,
$p=2 ,\ldots, P$, and derivatives of the expectation function with
respect to the new parameters can be determined simply by using the
chain rule.
To determine a likelihood interval for the response at a particular
point we find the values of $\phi_{1}$ such that
$\tau ( \phi_1 )=\pm\tNP$, and to determine a
likelihood band for the fitted response function at any $\bx$, we find the
values of $\phi_{1}$ such that $\tau ( \phi_1 )=\pm\FPNP$.

EXAMPLE:purolbad Puromycin likelihood band
%%%\input{purolband}

EXAMPLE:bodlbad BOD likelihood band
%%%\input{bodlband}

\section{Using Profile Plots to Determine a Good Parameterization}
\index{reparameterization}

In addition to providing exact likelihood intervals for each parameter,
the profile $t$ plots provide valuable information on the 
behavior of each parameter estimate, as do the profile trace
plots on the pair-wise behavior of the parameter estimates.
This is of great use in determining parametrizations that should
be used for future data sets with the same model function.
Such parametrizations can also be used to provide accurate linear
approximation marginal and joint parameter regions, and so obviate the
need for plots.  Reparametrization can also be used to accelerate convergence.

To determine useful parameterizations we use the technique introduced in
Section 1.5.4 for straightening a curve.  This is best explained by means of examples.
More applications are given in later Chapters.

EXAMPLE:bodrepar BOD reparameterization
%%%\input{bodrepar}

EXAMPLE:iso3 isomerization reparameterization
%%%\input{iso3}

More applications to demonstrate how to use profiling for fun and profit are given in later Chapters.

\section{Bayes Regions}

Inferences about nonlinear models using the Bayesian approach
\index{Bayes!inference for nonlinear models}
involve the same difficulties as the likelihood approach, with the
additional complexity of choosing a prior density for the parameters.
\subsection{Choice of Bayes Prior on the Parameters}
\index{prior density!for nonlinear model}

In the Bayesian analysis of a linear regression model, described
in Section 1.3.3, a prior density
\begin{displaymath}
p ( \bbeta , \sigma )  \propto  \sigma^{-1}
\end{displaymath}
is often used.
This does not correspond to an actual probability density, since
the integral of this density over the parameter $\sigma$ or $\bbeta$
is infinite, but the use of such ``uninformative''
or ``improper''
priors can be justified because the posterior density obtained by
multiplying the prior by the likelihood function {\em is\/}
a proper density.
That is, the posterior density has a finite integral over all possible
parameter values and can be normalized so that the integral is unity.
This follows because, for linear models, whenever
$\norm \bbeta \norm  \to  \infty$,
$\norm \boeta ( \bbeta ) \norm  \to  \infty$, and since $\by$ is fixed, this
means that $\norm \bz \norm \to \infty$, so $l ( \bbeta , \sigma | \by ) \to 0$.
[Technically, more is required for the finite integral:
we must have $l ( \bbeta , \sigma | \by )$ going to zero
``quickly'', which it does.]

An uninformative prior can usually be considered to be the limit of
proper priors that are more and more diffuse
(for example, a limit of multivariate
normal priors on $\bbeta$ with variance--covariance matrices
consisting of a fixed matrix multiplied by a factor that approaches infinity).
In this limiting process, quantities calculated from the
posterior densities, such as highest posterior density
(HPD) regions, approach finite limits smoothly.
However, for nonlinear models it is not always true that locally
uniform priors produce proper posterior densities.
Nonlinear models frequently have asymptotes, so
$\norm \btheta \norm \to \infty$ does not imply that
$\norm \boeta ( \btheta ) \norm \to \infty$ or $l ( \btheta , \sigma | \by ) \to 0$.
[For example, as $\theta  \to  \infty$ the Rumford model approaches a
finite limit, the point $(60 ,..., 60 ) \trans$.]
This means that when an improper prior on the parameters
is used, the posterior density will also be improper.
Even if one regards the uninformative prior as being the limit of a
sequence of proper prior densities, the situation is not improved
because the properties of the posterior density do not approach a
finite limit satisfactorily.
For example, if one were to apply a uniform prior density on
$\theta$ for the Rumford model over the interval $0 \le \theta  k$
and let $k$ approach infinity while calculating a 95\% HPD interval
for each value of $k$, the right hand end points of the
HPD intervals would go to infinity.

Consideration of the Rumford example enables us to see how to avoid
obtaining improper posterior densities:
instead of putting a locally uniform prior on the parameter
space, we should put a locally uniform prior %
{\em on the expectation surface }
\index{expectation surface!prior density}
\index{prior density!on expectation surface}
to represent an uninformative prior \cite{bate:1978}.
%\glossary{ Bates, D.M.}
For linear models, a locally uniform prior in the
parameter space produces a locally uniform prior on the expectation plane,
so this prior is consistent with standard practice for the linear model.
For a nonlinear model, the uninformative prior on $\btheta$ is
proportional to the Jacobian of the mapping to the expectation plane,
\index{Jacobian}
so we set
\begin{eqnarray*}
  p ( \btheta )&\propto&\frac{d A}{d\btheta}\\
  &=&| \bV \trans \bV |^{{1/2}}
\end{eqnarray*}
where $dA$ represents an element of area on the expectation surface.

For the Rumford example, the expectation surface is finite, so a uniform
prior density over the expectation surface will be a proper prior
density and hence the posterior density will be a proper density.
When the expectation surface has infinite extent, so the
uninformative prior is not a proper density, the
likelihood will force the posterior density to be
proper because the surface extends to infinity.
(Mathematically it is possible to get surfaces of infinite extent
which are restricted to a finite region in the response space, but
such pathological cases would not occur as expectation surfaces.)

We could also justify a locally uniform prior on the expectation surface
by arguing that prior ignorance about the responses corresponds to a
locally uniform prior on the sample space for the responses.
This induces a locally uniform prior over the
expectation surface as the prior for the model parameters $\btheta$.
Note that this choice makes the prior
{\em independent of the parametrization\/}
used in the model function,
since the Jacobian for the transformation cancels out.

EXAMPLE:puro17 Puromycin Bayes prior
%%%\input{puro14}

EXAMPLE:bod11 BOD Bayes prior
%%%\input{bod11}

To complete the prior density, we must specify the joint prior
for $\btheta$ and $\sigma$.
Following \citeasnoun{box:tiao:1973}, we choose independent priors
%\glossary{ Box, G.E.P.}
%\glossary{ Tiao, G.C.}
for $\btheta$ and $\sigma$ with $p ( \sigma ) \propto \sigma^{-1}$,
so the joint prior density is
\begin{eqnarray}
  p ( \btheta , \sigma )&\propto&p ( \btheta )p ( \sigma )\\
  &=&| \bV \trans \bV |^{{1/2}} \sigma^{-1}
  \label{eqn:jointprior}
\end{eqnarray}

It is noteworthy that the prior density does not depend on the
parametrization, but it is equally noteworthy that the prior
depends on the design used by the experimenter.
This makes good sense, since no scientific experiment is done in
the complete absence of knowledge, and the prior subtly expresses
the current knowledge of the experimenter through the design.

To illustrate, suppose one wished to measure BOD
of a river, and the model function
$f = \theta_1 ( 1 - e^{ - \theta_2 x } )$ was considered
appropriate.
The BOD, and consequently the parameters, would
depend on the season of the year, the latitude of the river, the rate
of flow of the river, the number and types of sources of pollution
along the river, the type of river bed, and so on.
For fast-flowing northern mountain rivers with no immediate pollution
sources, the rate constant $\theta_{2}$ would likely be very small,
and so samples taken from the river should be analyzed at rather long
intervals and over a long time, say every three days for three weeks.
For a sluggish meandering river in an industrial area,
on the other hand, the rate constant $\theta_{2}$ would
likely be large, and so the samples should be analyzed at shorter
intervals and over a shorter time, say every twelve hours for five days.
These considerations clearly affect the design and, in turn, the prior
on the parameters.

\subsection{Joint HPD Regions}

After choosing a prior density, we form the posterior density by
multiplying the prior by the likelihood function.
Thus the posterior density, $p ( \btheta , \sigma | \by )$, becomes
\index{posterior density!for nonlinear model}
\begin{displaymath}
p ( \btheta , \sigma | \by )\propto| \bV \trans \bV |^{{1/2}}
\sigma^{-(N+1)}  \exp \left( - \frac{S(\btheta)}{2\sigma^2}\right)
\end{displaymath}
where $S ( \btheta )$ is the residual sum of squares at $\btheta$.
As shown in \citeasnoun{box:tiao:1973}, the marginal posterior density of
%\glossary{ Box, G.E.P.}
%\glossary{ Tiao, G.C.}
$\btheta$ is obtained by integrating the joint posterior over the
nuisance parameter $\sigma$
to yield
\begin{eqnarray}
  p(\btheta|\by)&=&\int_{0}^{\infty}  p ( \btheta , \sigma | \by )  d \sigma\\
  &\propto&| \bV \trans \bV |^{{1/2}} [ S ( \btheta ) ]^{-N/2}
  \label{eqn:2.9d}
\end{eqnarray}
An HPD region will be bounded by a contour of this posterior
\index{highest posterior density (HPD)!region for nonlinear model}
density function, or equivalently, by a contour of
\begin{displaymath}
\frac{S(\btheta)}{|\bV\trans\bV|^{1/N}}
\end{displaymath}

To assign a probability value to the region,
we must determine the level of contour.
The exact method for determining a $(1 - \alpha)$ HPD region
is to integrate (\ref{eqn:2.9d}) over all possible values of $\btheta$,
obtain the constant of proportionality, and then integrate
the normalized posterior within contours of the posterior density until
we find the one with the required probability content.

All the integrations in this exact procedure make it too computationally
intensive for general use, but fortunately there is a convenient
approximation to the probability content of a contour.
For any expectation surface, a set of {\em geodesic\/}
parameters exists,
say $\bphi$, for which the prior density will be
approximately constant near $\hat{\bphi}$.
That is,
\begin{equation}
  | \bV_{\phi} \trans \bV_{\phi} |  \approx  { {\rm constant}}
  \label{eqn:2.9e}
\end{equation}
where
\begin{displaymath}
\bV_{\phi}=\frac{d\boeta}{d\bphi\trans}
\end{displaymath}
If the expectation surface is perfectly flat or, more generally,
if it has zero Gaussian curvature everywhere \cite{onei:1966},
%\glossary{ Oneill, B.}
(\ref{eqn:2.9e}) is an equality.

Besides having a locally uniform prior density, the $\bphi$
parameters also have an easily expressed likelihood function,
since when
the expectation surface is reasonably flat over the region of
nonnegligible likelihood, the sum of squares function
$S_{\phi} ( \bphi )$ is quadratic in $\bphi$.
That is,
\begin{eqnarray*}
  p_{\phi} ( \bphi | \by )&\propto&[ S_{\phi} ( \bphi ) ]^{-N/2}\\
  &\approx& \left[ S_{\phi} ( \hat{\bphi})  + \frac{1}{2}
  ( \bphi-\hat{\bphi})\trans
  \left.\frac{\partial^2S_{\phi}}{\partial\bphi\partial\bphi
  \trans} \right|_{\hat{\phi}}(\bphi-\hat{\bphi})\right]^{-N/2}  
\end{eqnarray*}
which is in the form of a multivariate T density
\index{T distribution!multivariate}
\cite{box:tiao:1973}.
%\glossary{ Box, G.E.P.}
%\glossary{ Tiao, G.C.}
Thus, an approximate $1-\alpha$ HPD region consists of all
values of $\bphi$ enclosed by the contour $S_{\phi} ( \bphi )$
determined by
\begin{displaymath}
\frac{[S_{\phi}(\bphi)-S_{\phi}(\hat{\bphi})]/P}
{{S_{\phi}(\hat{\bphi})}/(N-P)}\le\FPNP
\end{displaymath}

In terms of $\btheta$, the HPD region becomes a contour in
${S ( \btheta )} / {| \bV \trans \bV |^{1/N}}$, and so the approximate
$ (1 - \alpha)$ HPD region in $\btheta$ is bounded by the contour
\begin{displaymath}
\frac{S(\btheta)}{| \bV \trans \bV |^{1/N}} =
\frac{S(\hat{\btheta})}{|\hat{\bV}\trans\hat{\bV}|^{1/N}}
\left[ 1 + \frac{P}{N-P}\FPNP\right]
\end{displaymath}
Computationally, it is more convenient to determine
contours of ${ | \bV \trans \bV |^{1/N}} / {S ( \btheta ) }$, since
sometimes $| \bV \trans \bV | = 0$ on the boundary of the parameter
region, as in Example BOD \ref{bod:1c}.

EXAMPLE:puro18 Puromycin joint posterior
%%%\input{puro18}

EXAMPLE:bod12 BOD joint posterior
%%%\input{bod12}

Sketches could be made of the pairwise projections of HPD regions
using the same approach as for likelihood regions.
Instead of minimizing the conditional sum of squares, however,
we determine the posterior profile traces by minimizing
\index{profile trace!posterior}
\index{posterior density!profile trace}
$ { S ( \btheta ) }  / { | \bV \trans \bV |^{ 1 / N } } $.
Minimizing this is much more difficult than minimizing the profile sum
of squares, but fortunately the {\em locus\/} of the minima, say
$\btheta_{-p}^* ( \theta_p ) $ (i.e., the posterior profile trace)
will be essentially the same as
$\tilde \btheta_{-p} ( \theta_p ) $ (the likelihood trace)
because the term $| \bV \trans \bV |^{ 1 / N } $
will usually vary more slowly than the term $ S ( \btheta ) $.
It is therefore easy to evaluate the posterior density along the
likelihood trace and so generate sketches of the projections of
HPD regions using the methods of Appendix 6.

In theory, Bayesian marginal inferences are straightforward:
the posterior density is simply integrated over the nuisance
parameters, as in eliminating $\sigma$.
If $\theta_{1}$ is the single parameter of interest
and $\btheta_{-1}$ represents the nuisance parameters, the
marginal density for $\theta_{1}$ is
\begin{displaymath}
p_{{\theta}_1} ( \theta_1 )  \propto
\int  p_{\theta} ( \theta_1 , \btheta_{-1} \trans )
d \btheta_{-1}
\end{displaymath}
This method of eliminating components of $\btheta$ would generally
require a prohibitive amount of numerical integration, so approximations
based on the density conditional on the parameter of interest are used.
A first approximation is to use the analogue of the profile likelihood,
\begin{displaymath}
\int_{\theta} p( \theta_1 , \btheta_{-1} \trans ) 
d \btheta_{-1}\propto
p_{\theta} ( \theta_1 , \btheta_{-1}^* ( \theta_1 ))
\end{displaymath}
where $\btheta_{-1}^* ( \theta_1 )$ is the value which maximizes
$p_{\theta} ( \theta_1 , \btheta_{-1} )$ over $\btheta_{-1}$ for
that value of $\theta_{1}$.
That is, the marginal density for $\theta_{1}$
is assumed to be proportional to the maximum
value of the conditional density on $\theta_{1}$.
To obtain more accurate intervals,
a quadratic approximation could be used in which the integral
of the joint density is replaced by the product of its maximum
value and a measure of its spread at the maximum,
as discussed in \citeasnoun{tier:kada:1986}.
%\glossary{ Tierney, L.}
%\glossary{ Kadane, J.B.}

\section{Exact Sampling Theory Confidence Regions}

Sampling theory methods for linear regression can be extended to provide
joint confidence regions for parameters in nonlinear regression models.
\index{confidence!region for nonlinear model}
We only do this for completeness, however, because
{\em we do not recommend the approach}.

The method involves hypotheses of the form
\begin{displaymath}
{\rm H}_{0:}  \btheta = \btheta_0
\end{displaymath}
versus
\begin{displaymath}
{\rm H}_{{{\rm }} A}:  \btheta \ne \btheta_0
\end{displaymath}
where, as in the linear model, the test is based on the relative lengths
of tangential and orthogonal components of the residual vector
$\bz_0 = \by - \boeta ( \btheta_0 )$.
For linear models, the tangent plane is independent of $\btheta$
and the length of the orthogonal component of the residual vector
is fixed.
For nonlinear models, however,
the tangent plane changes with the value of $\btheta$ and so does
the length of the orthogonal component.

Even with a nonlinear model, this test provides a
locally most powerful, unbiased test of the hypothesis,
and under
the assumptions on the model, the confidence region
\begin{equation}
  \frac{\norm\bQ_1\trans(\btheta)\bz(\btheta)\norm^2/P}{
  \norm \bQ_2 \trans ( \btheta ) \bz ( \btheta ) \norm^2 / ( N -P )} 
  \le\FPNP
  \label{eqn:2.12c}
\end{equation}
is an exact $ (1 - \alpha)$ confidence region.
In (\ref{eqn:2.12c}), $\bQ_1 ( \btheta )$ and $\bQ_2 ( \btheta )$ are
the orthogonal parts of the
$QR$ decomposition of $\bV ( \btheta )$, the derivative matrix
evaluated at $\btheta$.
When there is an independent
variance estimate $s_r^{2}$ with $\nu_{r}$
degrees of freedom, an alternative form of the confidence region is
\begin{displaymath}
\norm \bQ_1 \trans ( \btheta ) \bz ( \btheta ) \norm^2 
\le P s_r^2 F ( P, \nu_r ; \alpha )
\end{displaymath}
EXAMPLE:puro19 Puromycin confidence regions
%%%\input{puro19}

EXAMPLE:bod13 BOD  confidence regions
%%%\input{bod13}

As pointed out in \citeasnoun{beal:1960}, sampling theory
%\glossary{ Beale, E.M.L.}
confidence regions have undesirable properties
because they are determined by values of a
ratio in which both the numerator and the denominator vary with $\btheta$.
The ratio in (\ref{eqn:2.12c}) can be small because the
tangential component is small or
because the orthogonal component is large.
When the expectation surface bends,
the ratio often falls below the critical value for points which are
very far away from the least squares values because the tangent
plane has tilted.

EXAMPLE:rum7 Rumford Confidence intervals
%%%\input{rum7}


We see that even for a simple 1-parameter model, exact confidence
regions can consist of disjoint portions
which include parameter values
whose residual vectors are much longer than the residual
vector at $\hat \btheta$.
Moreover, even without producing confidence regions in disjoint
portions, the method is subject to enclosing clearly
inappropriate parameter values in the confidence sets.

Marginal inferences are difficult to formulate using sampling theory
because the method for constructing joint or marginal confidence regions
for the parameters in a linear model does not generalize to
nonlinear models except when conditional
linearity can be exploited \cite{hami:1986}.
%\glossary{ Hamilton, D.C.}
As described in Section 1.2.3
for linear models, the method involves decomposing
the residual vector for a representative point on the subplane
defined by $\beta_2 = c$ into three components:  one
{\em orthogonal to\/} the expectation plane, one 
{\em in\/} the expectation plane and 
{\em orthogonal to\/} the subplane, and one {\em parallel to\/}
the subplane.
For a nonlinear model, both the expectation surface and
the subsurface generated by a constraint such as
$\theta_2 = c$ are nonlinear, and
there is no general decomposition giving three orthogonal
components.
In theory, marginal confidence regions could be computed, but a
reference distribution would have to be computed for each model,
each data set, and each parameter value, and so there would be
no straightforward way to obtain marginal
confidence regions using standard (F or $t$) distributions.

The only instance where the approach from linear models can be used
is when there are conditionally linear parameters.
If the model reduces to a linear model when a subset of the
parameters is held fixed, then exact marginal confidence regions
for the nonlinear parameters can be calculated.
For example, the model
\begin{displaymath}
f( x ,  \btheta ) = \theta_1 +  \theta_2 e^{ \theta_3 x }
\end{displaymath}
reduces to a linear model when $\theta_{3}$ is held fixed.
Thus confidence intervals for $\theta_{3}$ can be determined
using a $t$ distribution,
as described in \citeasnoun{halp:1963} and \citeasnoun{will:1962}.
%\glossary{ Halperin, M.}
%\glossary{ Williams, E.J.}

\section{Comparison of the Likelihood, Bayes, and Sampling Theory Approaches}
\sectionmark{Comparison of Approaches}

Four methods for summarizing joint and marginal inference regions
have been given.
In Chapter 2 we discussed linear approximations, and in this chapter
we presented profile $t$ and profile pair plots,
Bayes joint and marginal HPD procedures, and sampling theory procedures.
The linear approximation methods are the easiest to use, and
most nonlinear regression programs automatically provide the linear
approximation standard errors and estimate codependence matrix 
for the parameters.
The major disadvantage of linear approximations is that the validity
\index{linear approximation!to expectation function}
of the approximation over the region of interest is not known.
This approximation involves both the {\em planar\/}
assumption and the
\index{planar assumption}
\index{assumptions!planar}
{\em uniform coordinate\/}
\index{uniform coordinate!assumption}
\index{assumptions!uniform coordinate}
assumption, one or both of which could be invalid---as shown in the
next chapter, it is usually the uniform coordinate 
assumption.

The other approaches to inference produce joint regions
defined by contours.
Determining and displaying exact contours is generally too
expensive when $P > 2$,
but displaying profile $t$ plots, profile traces, and profile pair
sketches is eminently practical.
This approach requires a minor amount of extra
computation over the linear approximation approach but provides
valuable information on the behavior of the marginal and joint regions.
Fortunately, the calculations only require slight modifications of
standard nonlinear programs.
The confidence level associated with likelihood regions is
not well defined, but in
contrast to the linear approximation intervals, the likelihood regions
require only the planar assumption.

Similarly, examination of the approximate marginal and joint
HPD regions provides much more information than does
the linear approximation.
The Bayes prior moderates the tendency of likelihood contours to open
as the model function approaches an asymptote, making the Bayes
intervals more satisfactory.
The probability associated with the HPD intervals is
based on an approximation to the expectation surface that, like
the likelihood approach, only requires the planar assumption.
The Bayes approach has stronger justification for the
probability content or level of a region than the likelihood approach.

It is noteworthy that the prior density does not depend on the
parametrization, but it is equally noteworthy that the prior
depends on the design used by the experimenter.
As demonstrated in Section 6.2.1, however,
this makes good sense, since no scientific experiment is done in
the complete absence of knowledge, and the prior expresses
the current knowledge of the experimenter through the design.

The sampling theory approach, which uses the ratio of lengths of
components of the residual vector rather than the total length of
the residual vector to determine the region, can result in regions
which include inappropriate parameter values.
In addition, the difficulty of defining marginal regions for a
general case makes this approach completely unsuitable.

\section*{Summary}

In this Chapter we have presented 
procedures for computing and presenting exact likelihood, 
Bayesian, and sampling theory inference intervals.  
When a nonlinear regression is performed
we recommend producing the linear approximation
standard errors and parameter estimate codependencies, and profile $t$ plots, profile
traces, and possibly likelihood profile pair sketches.
For models that are used repeatedly, it is recommended that the profile
information be used to develop model parameterizations that yield
nearly straight profile t and trace plots so as to obviate the need for profiling.

In the next Chapter we discuss practical considerations in performing
a successful nonlinear estimation, and apply profiling in more examples.


% Local Variables: 
% mode: latex
% TeX-master: "nraia2"
% End: 
